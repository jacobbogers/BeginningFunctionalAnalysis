\documentclass[../../Solutions.tex]{subfiles}

\begin{document}

\begin{itemize}
	\item [3.1.1] Suppose $\mathcal{B}$ is countable.
		Then enumerate the elements according to the one-to-one correspondence with $\mathbb{N}$.
		\begin{equation*} \begin{split}
			a_1 & = \{a_{11},a_{12},a_{13},\dots\} \\
			a_2 & = \{a_{21},a_{22},a_{23},\dots\} \\
			a_3 & = \{a_{31},a_{32},a_{33},\dots\}
		\end{split} \end{equation*}
		Define a new sequence in $b \in \mathcal{B}$ as
		\begin{equation*} b_n = \begin{cases}
			1 & \text{if } a_{nn} = 0 \\
			0 & \text{if } a_{nn} = 1
		\end{cases} \end{equation*}
		But then $b$ cannot be represented in the enumeration above.
		Thus $\mathcal{B}$ cannot be countable.
	
	\item  [3.1.2] $\mathcal{B}_T$ is countable because each member $\{a_i\} \in \mathcal{B}_T$ can be identified with a terminating series
		$$ \{a_i\} \to \sum_{i=1}^\infty \frac{a_i}{2^i} $$
		which means each element of $\mathcal{B}_T$ is identified with a unique rational.
		Thus $\mathcal{B}_T$ is in one-to-one correspondence with a subset of $\mathbb{Q}$. \\
		Now, each number $r \in (0,1)$ can be written as a non-terminating binary expansion
		$$ r = \sum_{i=1}^\infty \frac{a_i}{2^i} \quad\text{where } a_i \in \{0,1\} $$
		Then these $\{a_i\}$ represent the members of $\mathcal{B}\setminus\mathcal{B}_T$ (because they are non-terminating).
		Thus $\mathcal{B}\setminus\mathcal{B}_T$ is uncountable which means $\mathcal{B}$ is also uncountable.
	
	\item [3.1.4] Let each member $r \in (0,1]$ be written in a binary expansion $r = 0.a_1a_2a_3\dots$.
		Define $\phi:(0,1]\to(0,1]$ as
		$$ \phi(0.a_1a_2a_3\dots) = 0.a_111a_211a_311\dots $$
		$\phi$ is one-to-one. Suppose $\phi(0.a_1a_2a_3\dots) = \phi(0.b_1b_2b_3\dots)$, then
		$$ 0.a_111a_211a_311\dots = 0.b_111b_211b_311\dots \quad\Longrightarrow\quad a_i = b_i \text{ for all } i \in \mathbb{N} $$
		Thus $0.a_1a_2a_3\dots = 0.b_1b_2b_3\dots$. \\
		Let $r \in \phi((0,1])$. Then $r = 0.r_111r_211r_311\dots$ which means
		$$ \lim_{n\to\infty} \frac{s_n(r)}{n} \to \infty $$
		So $r \in (0,1]\setminus S$.
		Thus $\phi((0,1]) \subseteq (0,1]\setminus S$, which means $(0,1]\setminus S$ is uncountable.
	
	\item [3.2.1] First we will prove a necessary set identity.
		\begin{equation*} \begin{split}
			A_1 \setminus \bigcup_{n=1}^\infty (A_1 \setminus A_n) & = A_1 \cap \left( \bigcup_{n=1}^\infty (A_1 \cap A_N^C) \right)^C \\
				& = A_1 \cap \bigcap_{n=1}^\infty (A_1^C \cup A_n) \\
				& = \bigcap_{n=1}^\infty (A_1 \cap (A_1^C \cup A_n)) \\
				& = \bigcap_{n=1}^\infty A_n
		\end{split} \end{equation*}
		Now let $\mathcal{R}$ be a $\sigma$-ring.
		Suppose $A_n \in \mathcal{R}$ for all $n \in \mathbb{N}$.
		Then $A_1 \setminus A_n \in \mathcal{R}$ for all $n \in \mathbb{N}$.
		Thus $\bigcup_{n=1}^\infty (A_1 \setminus A_n) \in \mathcal{R}$.
		Therefore
		$$ \bigcap_{n=1}^\infty A_n = A_1 \setminus \bigcup_{n=1}^\infty (A_1 \setminus A_n) \in \mathcal{R} $$
	
	\item [3.2.2] Let $\mu$ be a non-negative and additive function on a ring $\mathcal{R}$.
	\begin{enumerate}[(a)]
		\item Let $A,B \in \mathcal{R}$ with $A \subseteq B$, then
			$$ B = A \cup (B\setminus A) $$
			Thus, using additivity and non-negativity, we can conclude
			$$ \mu(A) \leq \mu(A)+\mu(B\setminus A) = \mu(A\cup(B\setminus A)) = \mu(B) $$
			Therefore $\mu$ is monotone.
		\item We can apply induction to part (a) above.
			Assume
			$$ \mu\Big(\bigcup_{k=1}^n A_k \Big) \leq \sum_{k=1}^n \mu(A_k) $$
			Then, using part (a) in the first inequality,
			\begin{equation*} \begin{split}
				\mu\Big(\bigcup_{k=1}^{n+1} A_k \Big) & \leq \mu\Big(\bigcup_{k=1}^{n} A_k \Big) + \mu(A_{n+1}) \\
					& \leq \sum_{k=1}^{n} \mu(A_k) + \mu(A_{n+1}) = \sum_{k=1}^{n+1} \mu(A_k)
			\end{split} \end{equation*}
			%MORE DETAIL
	\end{enumerate}
	
	\item [3.2.3] Assume $\mu$ is countably additive on the ring $\mathcal{R}$.
		First, we prove a helpful identity.
		If $A \subseteq B$ inside $\mathcal{R}$, then
		$$ \mu(A) + \mu(B\setminus A) = \mu(A \cup B\setminus A) = \mu(B) $$
		Thus
		$$ \mu(B\setminus A) = \mu(B) - \mu(A) $$
		Now, suppose $A_n, A \in \mathcal{R}$ satisfy
		$$ A_1 \subseteq A_2 \subseteq A_3 \subseteq \cdots \quad\text{and}\quad A = \bigcup_{n=1}^\infty A_n $$
		Define $B_1 = A_1$ and $B_n = A_n \setminus A_{n-1}$ for all $n \geq 2$.
		Then $B_n \in \mathcal{R}$, $A = \bigcup_{n=1}^\infty B_n$, and $B_i \cap B_j = \emptyset$ if $i \neq j$.
		Thus, for any $N \in \mathbb{N}$ (and using our helpful identity from above)
		\begin{equation*} \begin{split}
			\mu\Big(\bigcup_{n=1}^N B_n\Big) & = \sum_{n=1}^N \mu(B_n) \\
				& = \mu(A_1) + \mu(A_2 \setminus A_1) + \mu(A_3 \setminus A_2) + \cdots + \mu(A_N\setminus A_{N-1}) \\
				& = \mu(A_1) + \mu(A_2) - \mu(A_1) + \mu(A_3) - \mu(A_2) + \cdots + \mu(A_N) - \mu(A_{N-1}) \\
				& = \mu(A_N)
		\end{split} \end{equation*}
		Since $\mu$ is countably additive, we can exchange $\mu$ and the limit with respect to $N$.
		Thus,
		\begin{equation*} \begin{split}
			\mu(A) & = \mu\Big(\lim_{N\to\infty}\bigcup_{n=1}^N B_n\Big) \\
				& = \lim_{N\to\infty}\mu\Big(\bigcup_{n=1}^N B_n\Big) \\
				& = \lim_{N\to\infty}\mu(A_N)
		\end{split} \end{equation*}
		proving the desired conclusion.
	
\end{itemize}

\end{document}