\documentclass[../../Solutions.tex]{subfiles}

\begin{document}

\begin{itemize}
	\item [3.1.1] Suppose $\mathcal{B}$ is countable.
		Then enumerate the elements according to the one-to-one correspondence with $\mathbb{N}$.
		\begin{equation*} \begin{split}
			a_1 & = \{a_{11},a_{12},a_{13},\dots\} \\
			a_2 & = \{a_{21},a_{22},a_{23},\dots\} \\
			a_3 & = \{a_{31},a_{32},a_{33},\dots\}
		\end{split} \end{equation*}
		Define a new sequence in $b \in \mathcal{B}$ as
		\begin{equation*} b_n = \begin{cases}
			1 & \text{if } a_{nn} = 0 \\
			0 & \text{if } a_{nn} = 1
		\end{cases} \end{equation*}
		But then $b$ cannot be represented in the enumeration above.
		Thus $\mathcal{B}$ cannot be countable.
	
	\item  [3.1.2] $\mathcal{B}_T$ is countable because each member $\{a_i\} \in \mathcal{B}_T$ can be identified with a terminating series
		$$ \{a_i\} \to \sum_{i=1}^\infty \frac{a_i}{2^i} $$
		which means each element of $\mathcal{B}_T$ is identified with a unique rational.
		Thus $\mathcal{B}_T$ is in one-to-one correspondence with a subset of $\mathbb{Q}$. \\
		Now, each number $r \in (0,1)$ can be written as a non-terminating binary expansion
		$$ r = \sum_{i=1}^\infty \frac{a_i}{2^i} \quad\text{where } a_i \in \{0,1\} $$
		Then these $\{a_i\}$ represent the members of $\mathcal{B}\setminus\mathcal{B}_T$ (because they are non-terminating).
		Thus $\mathcal{B}\setminus\mathcal{B}_T$ is uncountable which means $\mathcal{B}$ is also uncountable.
	
	\item [3.1.4] Let each member $r \in (0,1]$ be written in a binary expansion $r = 0.a_1a_2a_3\dots$.
		Define $\phi:(0,1]\to(0,1]$ as
		$$ \phi(0.a_1a_2a_3\dots) = 0.a_111a_211a_311\dots $$
		$\phi$ is one-to-one. Suppose $\phi(0.a_1a_2a_3\dots) = \phi(0.b_1b_2b_3\dots)$, then
		$$ 0.a_111a_211a_311\dots = 0.b_111b_211b_311\dots \quad\Longrightarrow\quad a_i = b_i \text{ for all } i \in \mathbb{N} $$
		Thus $0.a_1a_2a_3\dots = 0.b_1b_2b_3\dots$. \\
		Let $r \in \phi((0,1])$. Then $r = 0.r_111r_211r_311\dots$ which means
		$$ \lim_{n\to\infty} \frac{s_n(r)}{n} \to \infty $$
		So $r \in (0,1]\setminus S$.
		Thus $\phi((0,1]) \subseteq (0,1]\setminus S$, which means $(0,1]\setminus S$ is uncountable.
	
	\item [3.2.1] First we will prove a necessary set identity.
		\begin{equation*} \begin{split}
			A_1 \setminus \bigcup_{n=1}^\infty (A_1 \setminus A_n) & = A_1 \cap \left( \bigcup_{n=1}^\infty (A_1 \cap A_N^C) \right)^C \\
				& = A_1 \cap \bigcap_{n=1}^\infty (A_1^C \cup A_n) \\
				& = \bigcap_{n=1}^\infty (A_1 \cap (A_1^C \cup A_n)) \\
				& = \bigcap_{n=1}^\infty A_n
		\end{split} \end{equation*}
		Now let $\mathcal{R}$ be a $\sigma$-ring.
		Suppose $A_n \in \mathcal{R}$ for all $n \in \mathbb{N}$.
		Then $A_1 \setminus A_n \in \mathcal{R}$ for all $n \in \mathbb{N}$.
		Thus $\bigcup_{n=1}^\infty (A_1 \setminus A_n) \in \mathcal{R}$.
		Therefore
		$$ \bigcap_{n=1}^\infty A_n = A_1 \setminus \bigcup_{n=1}^\infty (A_1 \setminus A_n) \in \mathcal{R} $$
	
	\item [3.2.2] Let $\mu$ be a non-negative and additive function on a ring $\mathcal{R}$.
	\begin{enumerate}[(a)]
		\item Let $A,B \in \mathcal{R}$ with $A \subseteq B$, then
			$$ B = A \cup (B\setminus A) $$
			Thus, using additivity and non-negativity, we can conclude
			$$ \mu(A) \leq \mu(A)+\mu(B\setminus A) = \mu(A\cup(B\setminus A)) = \mu(B) $$
			Therefore $\mu$ is monotone.
		\item We can apply induction to part (a) above.
			Assume
			$$ \mu\Big(\bigcup_{k=1}^n A_k \Big) \leq \sum_{k=1}^n \mu(A_k) $$
			Then, using part (a) in the first inequality,
			\begin{equation*} \begin{split}
				\mu\Big(\bigcup_{k=1}^{n+1} A_k \Big) & \leq \mu\Big(\bigcup_{k=1}^{n} A_k \Big) + \mu(A_{n+1}) \\
					& \leq \sum_{k=1}^{n} \mu(A_k) + \mu(A_{n+1}) = \sum_{k=1}^{n+1} \mu(A_k)
			\end{split} \end{equation*}
			%MORE DETAIL
	\end{enumerate}
	
	\item [3.2.3] Assume $\mu$ is countably additive on the ring $\mathcal{R}$.
		First, we prove a helpful identity.
		If $A \subseteq B$ inside $\mathcal{R}$, then
		$$ \mu(A) + \mu(B\setminus A) = \mu(A \cup B\setminus A) = \mu(B) $$
		Thus
		$$ \mu(B\setminus A) = \mu(B) - \mu(A) $$
		Now, suppose $A_n, A \in \mathcal{R}$ satisfy
		$$ A_1 \subseteq A_2 \subseteq A_3 \subseteq \cdots \quad\text{and}\quad A = \bigcup_{n=1}^\infty A_n $$
		Define $B_1 = A_1$ and $B_n = A_n \setminus A_{n-1}$ for all $n \geq 2$.
		Then $B_n \in \mathcal{R}$, $A = \bigcup_{n=1}^\infty B_n$, and $B_i \cap B_j = \emptyset$ if $i \neq j$.
		Thus, for any $N \in \mathbb{N}$ (and using our helpful identity from above)
		\begin{equation*} \begin{split}
			\mu\Big(\bigcup_{n=1}^N B_n\Big) & = \sum_{n=1}^N \mu(B_n) \\
				& = \mu(A_1) + \mu(A_2 \setminus A_1) + \mu(A_3 \setminus A_2) + \cdots + \mu(A_N\setminus A_{N-1}) \\
				& = \mu(A_1) + \mu(A_2) - \mu(A_1) + \mu(A_3) - \mu(A_2) + \cdots + \mu(A_N) - \mu(A_{N-1}) \\
				& = \mu(A_N)
		\end{split} \end{equation*}
		Since $\mu$ is countably additive, we can exchange $\mu$ and the limit with respect to $N$.
		Thus,
		\begin{equation*} \begin{split}
			\mu(A) & = \mu\Big(\lim_{N\to\infty}\bigcup_{n=1}^N B_n\Big) \\
				& = \lim_{N\to\infty}\mu\Big(\bigcup_{n=1}^N B_n\Big) \\
				& = \lim_{N\to\infty}\mu(A_N)
		\end{split} \end{equation*}
		proving the desired conclusion.
	
	\item [3.2.7]
	\begin{enumerate}[(a)]
		\item %DRAW A & B
			We calculate $D(A,B)$ as the area of $S(A,B)$.
			$$ D(A,B) = m([0,1]\times[0,1])+m([1,4]\times[1,2])+m([0,4]\times[2,10]) = 36 $$
		\item First, we show that $S(A,B) = S(B,A)$.
			$$ S(A,B) = (A\setminus B)\cup(B\setminus A) = (B\setminus A)\cup(A\setminus B) = S(B,A) $$
			Then the desired conclusion follows quickly.
			$$ D(A,B) = m^*(S(A,B)) = m^*(S(B,A)) = D(B,A) $$
		\item $D(A,B) = 0$ does \emph{not} imply that $A = B$. For example in $\R$, $D([0,1],[0,1)) = 0$ but $[0,1] \neq [0,1)$.
		\item Let $x \in A\setminus C$.
			Then $x \in A$ and $x \not\in C$.
			If $x \in B$, then $x \in B\setminus C$.
			If $x \not\in B$, then $x \in A\setminus B$.
			Thus $x \in (A\setminus B)\cup(B\setminus C)$.
			Similarly, if $x \in C\setminus A$, $x \in (C\setminus B)\cup(B\setminus A)$.
			Then we can conclude
			$$ x \in (A\setminus C)\cup(C\setminus A) \Longrightarrow x \in (A\setminus B)\cup(B\setminus C)\cup(C\setminus B)\cup(B\setminus A) = (A\setminus B)\cup(B\setminus A)\cup(B\setminus C)\cup(C\setminus B) $$
			Which then (applying the definition of symmetric difference) means
			$$ S(A,C) \subseteq S(A,B)\cup S(B,C) $$
			Thus, since $m^*$ is monotone and finitely additive,
			\begin{equation*} \begin{split}
				D(A,C) & = m^*(S(A,C)) \\
					& \leq m^*(S(A,B)\cup S(B,C)) \\
					& = m^*(S(A,B))+m^*(S(B,C)) \\
					& = D(A,B)+D(B,C)
			\end{split} \end{equation*}
		\item We can conclude this containment using basic set operations.
			\begin{equation*} \begin{split}
				(A_1 \cup A_2)\setminus(B_1\cup B_2) & = (A_1\cup A_2)\cap(B_1^C\cap B_2^C) \\
					& = (A_1\cap B_1^C \cap B_2^C)\cup(A_2\cap B_1^C\cap B_2^C) \\
					& \subseteq (A_1 \cap B_1^C)\cup(A_2\cap B_2^C) \\
					& = (A_1\setminus B_1)\cup(A_2\setminus B_2)
			\end{split} \end{equation*}
			which means
			$$ S(A_1\cup A_2,B_1\cup B_2) \subseteq S(A_1,B_1)\cup S(A_2,B_2) $$
			Thus, since $m^*$ is monotone and finitely additive,
			\begin{equation*} \begin{split}
				D(A_1\cup A_2,B_1\cup B_2) & = m^*(S(A_1\cup A_2,B_1\cup B_2)) \\
				& \leq m^*(S(A_1,B_1)\cup S(A_2,B_2)) \\
				& = m^*(S(A_1,B_1))+m^*(S(A_2,B_2)) \\
				& = D(A_1,B_1)+D(A_2,B_2)
			\end{split} \end{equation*}
	\end{enumerate}
	
	\item [3.2.8] We will not go through displaying that the binary operations defined in this way follow all the requirements of a commutative ring.
	(Associativity and Distributivity are particularly tedious tasks in set manipulation)
	However, we would like to point out a few interesting notes.
	\begin{itemize}
		\item The additive identity in this ring is the empty set because $S(A,\emptyset) = A$ for all $A$.
		\item The additive inverse of each element is itself because $S(A,A) = \emptyset$ for all $A$.
		\item The multiplicative identity in this ring is $\R^n$ because $A\cap \R^n = A$ for all $A$.
		\item No element has a multiplicative inverse (expect the identity) because $A\cap B \subseteq A$ for all $A,B$.
		\item Moreover, each element (expect the $\R^n$) has several (indeed infinite) zero divisors.
			Any $B \subseteq A^C$ is a zero divisor of $A$ because $A\cap B \subseteq A\cap A^C = \emptyset$.
	\end{itemize}
	These notes show that $2^{(\R^n)}$ under intersection and symmetric difference is a commutative ring with unity (but not an integral domain).
	
	\item [3.2.9]
	\begin{enumerate}[(a)]
		\item Since $\R^n$ is finite-dimensional, we can use any norm we wish (because all norms are equivalent).
			We choose $\norm{\cdot}_\infty$ because it has $r$-balls that are equivalent to intervals as defined in the text.
			Then, if $x \in \R^n$ and $r > 0$, $B_r(x) \in \mathcal{E} \subseteq \mathcal{M}_F$.
			Let $U \subseteq \R^n$ be open.
			Define $U_Q = U\cap\mathbb{Q}^n$.
			For each $q \in U_Q$, define
			$$ r_q = \sup\{r > 0 : B_r(q) \subseteq U \} $$
			Then $B_{r_q}(q) \subseteq U$ for each $q$ and
			$$ \bigcup_{q \in U_Q} B_{r_q}(q) \subseteq U $$
			Now let $x \in U$.
			Then there exists $r > 0$ such that $B_r(x) \subseteq U$.
			Choose $q \in \mathbb{Q}^n$ such that
			$$ \norm{x-q}_\infty < \frac{r}{2} $$
			Then
			$$ x \in B_{r/2}(q) \subseteq B_{r_q}(q) \subseteq \bigcup_{q \in U_Q} B_{r_q}(q) $$
			showing that
			$$ U \subseteq \bigcup_{q \in U_Q} B_{r_q}(q) $$
			Combining these two containments, we obtain
			$$ U = \bigcup_{q \in U_Q} B_{r_q}(q) $$
			Therefore, since $U_Q \subseteq \mathbb{Q}^n$ is countable, we have written $U$ as a countable union of members of $\mathcal{M}_F$, meaning $U \in \mathcal{M}$.
		\item Let $F \subseteq \R^n$ be closed.
			Then $F^C$ is open and thus (by part (a)) $F^C \in \mathcal{M}$.
			Since $\mathcal{M}$ is a $\sigma$-ring, $F = (F^C)^C \in \mathcal{M}$.
		\item Let $\{E_i\}_{i=1}^\infty$ be a collection of open or closed sets.
			Then by parts (a) and (b), each $E_i$ is in $\mathcal{M}$, and since $\mathcal{M}$ is a $\sigma$-ring,
			$$ \bigcup_{i=1}^\infty E_i \in \mathcal{M} $$
			Because complements of sets in $\mathcal{M}$ are in $\mathcal{M}$,
			$$ \bigcap_{i=1}^\infty E_i = \Big(\bigcup_{i=1}^\infty E_i^C\Big)^C \in \mathcal{M} $$
			Thus countable unions and intersections of open and closed sets are in $\mathcal{M}$.
	\end{enumerate}
	
	\item [3.2.10] The usual definition of the Cantor set is
		$$ C = \bigcap_{i=1}^\infty \Big( \bigcup_{j=1}^{2^i} I^i_j \Big) $$
		where the $I^i_j \subseteq [0,1]$ are the intervals where the middle third is removed from the $i$th set to produce the $(i+1)$th step.
		These $I^i_j$ are disjoint in the $j$ coordinate, specifically
		$$ I^i_m \cap I^i_n = \emptyset \quad\text{if } m \neq n $$
		Thus $C$ is a countable intersection of sets that are finite unions of disjoint intervals, which means $C \in \mathcal{M}$. \\
		Since $C$ has no intervals contained in it, $C$ has zero measure.
	
	\item [3.3.1] Let $a \in \R$ be given.
		The set $(a,\infty)\subseteq\R$ is open.
		Since $f:\R^n\to\R$ is continuous,
		$$ f^{-1}((a,\infty)) = \{x : f(x) > a \} \R^n $$
		is open and therefore measurable by 3.2.9(a).
		Since $a$ is arbitrary, $f$ is a measurable function.
	
	\item [3.3.2] Let $A \subseteq \R^n$ be not measurable (exists by the Vitali Theorem).
		Then define $f:\R^n\to\R$ by
		\begin{equation*}
			f(x) = \begin{cases}
				1 & \text{if } x \in A \\
				-1 & \text{otherwise}
			\end{cases}
		\end{equation*}
		Then $A = \{ x : f(x) > 0 \}$ is not measurable because $A$ is not measurable.
		Thus $f$ is not measurable.
		But $|f|$ is a constant function, so it is measurable.
	
	\item [3.3.3] Characteristic functions are integrable if and only if the set it characterizes is measurable.
		The characteristic function of the rationals is integrable because $\mathbb{Q}\cap[0,1]$ is measurable.
		$$ \int_{[0,1]} \chi_{\mathbb{Q}} dm = m([0,1]\cap\mathbb{Q}) = 0 $$
	
	\item [3.3.4] Assume $f$ is measurable.
		Then $L_a = \{x : f(x) < a \}$ and $G_a = \{x : f(x) > a \}$ are measurable for all $a$.
		Let $a \in \R$ be given. We have
		$$ |f(x)| > a \iff f(x) > a \text{  or  } f(x) < -a $$
		Then
		$$ \{x : |f(x)|>a\} = \{x : f(x) > a \}\cup\{x : f(x) < -a\} = G_a\cup L_{-a} $$
		which is measurable because the union of two measurable sets is measurable.
		Since $a$ is arbitrary, $|f|$ is measurable.
		
	\item [3.3.7] By definition, $f \in \mathcal{L}(\R^n)$ if and only if
		$$ \int_{\R^n} f_+ dm < \infty \text{  and  } \int_{\R^n} f_- dm < \infty $$
		Since both of these integrals are finite, there sum is finite, so (using Theorem 3.11(a))
		$$ \int_{\R^n} |f| dm = \int_{\R^n} (f_+ + f_-) dm = \int_{\R^n} f_+ dm + \int_{\R^n} f_- dm < \infty $$
		which proves the desired conclusion.
	
	\item [3.4.1] Define $E_k = [-1/2k,1/2k] \subseteq \R$ and $s_k:[-1,1]\to\R$ as
		$$ s_k(x) = k\chi_{E_k}(x) $$
		Then 
		$$ \int_{[-1,1]} s_k dm = 1 \quad\text{for all } k \in \mathbb{N} $$
		which means $\liminf_{k\to\infty} \int_{[-1,1]} s_k dm = 1$.
		Also $s = \liminf_{k\to\infty} s_k$ is equal to the zero function almost everywhere, which means
		$$ \int_{[-1,1]} s dm = 0 $$
		Thus
		$$ \int_{[-1,1]} s dm = 0 < 1 = \liminf_{k\to\infty} \int_{[-1,1]} s_k dm $$
		which is an example of strict inequality holding in Fatou's Lemma.
	
	\item [3.4.2] Use the function sequence defined above.
		It is shown in 3.4.1 that the limit of integrals of this sequence is not equal to the integral of the limit.
		This sequence does not have an integrable function $g$ that bounds it.
	
	\item [3.6.3] Let $y \geq 0$ and $1 < p < \infty$ be fixed.
		Define $f:[0,\infty]\to\R$ as
		$$ f(x) = xy - \frac{x^p}{p} $$
		Then $f'(x) = y - x^{p-1}$, which is zero when $x = y^{1/(p-1)}$.
		Thus $f$ attains a maximum at $y^{1/(p-1)}$ meaning $f(x) \leq f(y^{1/(p-1)})$ for all $x \geq 0$.
		Thus
		$$ xy - \frac{x^p}{p} \leq y^{1+\frac{1}{p-1}} - \frac{1}{p}y^{\frac{p}{p-1}} $$
		Rearranging (noting that the H\"older conjugate of $p$ is given by $1/q = 1-1/p$) gives
		$$ xy \leq \frac{x^p}{p} + \left(1-\frac{1}{p}\right)y^{\frac{1}{1-1/p}} = \frac{x^p}{p} + \frac{y^q}{q} $$
	
	\item [3.6.4] Let $f \in L^p(\mu)$ and $1 < p < \infty$.
		Let $q$ be the H\"older conjugate of $p$.
		Then
		$$ (p-1)q = (p-1)\frac{p}{p-1} = p $$
		So following the definition of $\norm{\cdot}_p$ we have
		\begin{equation*} \begin{split}
			\norm{|f|^{p-1}}_q & = \Bigg(\int_{X} |f|^{(p-1)q} d\mu\Bigg)^\frac{1}{q} \\
				& = \Big(\int_{X} |f|^{(p-1)q} d\mu\Big)^\frac{p-1}{(p-1)q} = \big(\norm{f}_{(p-1)q}\big)^{p-1} \\
				& = \Big(\int_{X} |f|^p d\mu \Big)^\frac{p-1}{p} = \big(\norm{f}_p\big)^{p-1}
		\end{split} \end{equation*}
		Since $f,g \in L^p(\mu)$ means that $f+g \in L^p(\mu)$ we can conclude
		$$ \norm{|f+g|^{p-1}}_q = \big(\norm{f+g}_{(p-1)q}\big)^{p-1} = \big(\norm{f+g}_p\big)^{p-1} $$
	
	\item [3.6.5] Let $f,g \in L^\infty$ and suppose $M > 0$ satisfies $|f(x)| \leq M$ almost everywhere and $N > 0$ satisfies $|g(x)| \leq N$ almost everywhere.
		Then if $\lambda$ is a scalar, $|\lambda f(x)| = |\lambda||f(x)| \leq |\lambda|M$ almost everywhere, so $\lambda f \in L^\infty$.
		Also, $|f(x)+g(x)| \leq |f(x)|+|g(x)| \leq M+N$ almost everywhere, so $f+g \in L^\infty$.
		Thus $L^\infty$ is a linear space. \\
		Now we will show that $\norm{\cdot}_\infty$ satisfies the requirements to be a norm.
		\begin{enumerate}[(i)]
			\item $\norm{f}_\infty \geq 0$ for all $f$ because $|f(x)| \geq 0$ for all $f$ and all $x$.
			\item $\norm{f}_\infty = 0$ if and only if $0 \leq |f(x)| \leq 0$ almost everywhere which means $f$ is in the same equivalence class as the zero function.
			\item $\norm{\lambda f}_\infty = \inf\{M : |\lambda f(x)| \leq M\} = |\lambda|\inf\{M:|f(x)|\leq M\} = |\lambda|\norm{f}_\infty$
			\item $\norm{f+g}_\infty = \inf\{M:|f(x)+g(x)|\leq M\} \leq \inf\{M:|f(x)|+|g(x)|\leq M\} \leq \inf\{M:|f(x)|\leq M\} + \inf\{M:|g(x)|\leq M\} = \norm{f}_\infty + \norm{g}_\infty$
		\end{enumerate}
	
	\item [3.6.9] We will use the fact that if a function is both Lebesgue and Riemann integrable, then the value of the two integrals are the same.
	\begin{enumerate}[(a)]
		\item These functions are simple functions to integrate.
		\begin{equation*} \begin{split} 
			\norm{f}_p^p & = \int_{-1}^{1} |1+x|^p dx = \int_0^2 u^p du = \frac{2^{p+1}}{p+1} \\
			\norm{g}_p^p & = \int_{-1}^{1} |1-x|^p dx = \int_{-1}^1 |x-1|^p dx = \int_0^2 u^p du = \frac{2^{p+1}}{p+1} \\
			\norm{f+g}_p^p & = \int_{-1}^1 |1+x+1-x|^p dx = \int_{-1}^1 2^p dx = 2^{p+1} \\
			\norm{f-g}_p^p & = \int_{-1}^1 |1+x-1+x|^p dx = \int_{-1}^1 |2x|^p dx = \frac{2^{p+1}}{p+1}
		\end{split} \end{equation*}
		\item We can calculate this conclusion simply from the parallelogram equality.
			$$ \norm{f+g}_p^2 + \norm{f-g}_p^2 = 2\norm{f}_p^2 + 2\norm{g}_p^2 $$
			$$ \left(2^{p+1}\right)^{\frac{2}{p}} + \left(\frac{2^{p+1}}{p+1}\right)^{\frac{2}{p}} = 2\left(\frac{2^{p+1}}{p+1}\right)^{\frac{2}{p}} + 2\left(\frac{2^{p+1}}{p+1}\right)^{\frac{2}{p}} $$
			$$ 3\left(\frac{2^{p+1}}{p+1}\right)^{\frac{2}{p}} = \left(2^{p+1}\right)^{\frac{2}{p}} $$
			$$ 3\left(2^{p+1}\right)^{\frac{2}{p}} = \left(2^{p+1}(p+1)\right)^{\frac{2}{p}} $$
			$$ 3 = (p+1)^{\frac{2}{p}} $$
			This equality is satisfied with $p = 2$.
			$$ (2+1)^\frac{2}{2} = 3^1 = 3 $$
		\item Note that for $p \geq 1$,
			$$ \frac{\ln(p+1)}{p} > \frac{\ln(p+1)}{(p+1)\ln(p+1)} = \frac{1}{p+1} $$
			Let $f:[1,\infty)\to\R$ be defined as $f(p) = (p+1)^{2/p} - 3$.
			Then
			$$ f'(p) = \frac{2(p+1)^{2/p}}{p}\left(\frac{1}{p+1} - \frac{\ln(p+1)}{p}\right) $$
			and for $p \geq 1$, we have $f'(p) < 0$.
			Thus $f$ is a strictly decreasing function in $p$ and therefore can only have one zero: $p = 2$.
		\item Choose $r > 0$ and $c$ such that $B_r(c) \subseteq I$.
			Define $f,g:I\to\R$ by
			\begin{equation*} \begin{split}
				f(x) &= \begin{cases}
					1+\frac{1}{r}(x-c) & x \in B_r(c) \\
					0 & \text{otherwise}
				\end{cases} \\
				g(x) &= \begin{cases}
					1-\frac{1}{r}(x-c) & x \in B_r(c) \\
					0 & \text{otherwise}
				\end{cases}
			\end{split} \end{equation*}
			Then $f$ and $g$ yield the same results as those previously defined for $[-1,1]$.
			\begin{equation*} \begin{split} 
				\norm{f}_p^p & = \int_{c-r}^{c+r} |1+\frac{1}{r}(x-c)|^p dx = r\int_{-1}^{1} |1+x|^p dx = r\frac{2^{p+1}}{p+1} \\
				\norm{g}_p^p & = \int_{c-r}^{c+r} |1-\frac{1}{r}(x-c)|^p dx = r\int_{-1}^{1} |1-x|^p dx = r\frac{2^{p+1}}{p+1} \\
				\norm{f+g}_p^p & = \int_{c-r}^{c+r} |1+\frac{1}{r}(x-c)+1-\frac{1}{r}(x-c)|^p dx = \int_{c-r}^{c+r} 2^p dx = r2^{p+1} \\
				\norm{f-g}_p^p & = \int_{c-r}^{c+r} |1+\frac{1}{r}(x-c)-1+\frac{1}{r}(x-c)|^p dx = r\int_{-1}^{1} |2x|^p dx = r\frac{2^{p+1}}{p+1}
			\end{split} \end{equation*}
			which, combined with the parallelogram equality, produces the same restriction on $p$:
			$$ (p+1)^{\frac{2}{p}} = 3 $$
			Then parts (b) and (c) follow.
	\end{enumerate}
	
\end{itemize}

\end{document}