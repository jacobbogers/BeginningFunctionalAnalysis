\documentclass[../../Solutions.tex]{subfiles}
 
\begin{document}

\begin{itemize}
	\item [A.1]
		\begin{enumerate}[(a)]
			\item Multiplying the numerator and denominator by the complex conjugate produces the necessary simplification.
			$$ \frac{2-5i}{1+i}\frac{1-i}{1-i} = \frac{2-5i-2i-5}{1+1} = -\frac{3}{2}-\frac{7}{2}i $$
			
			\item We are able to show this equality simply by multiplying by $1$.
			$$ \frac{1}{i} = \frac{1}{i}\frac{i}{i} = \frac{i}{i^2} = \frac{i}{-1} = -i $$
		\end{enumerate}
		
	\item [A.2] For this exercise, we write $w = a+ib = Ae^{i\theta}$ and $z = c+id = Ce^{i\phi}$.
	The first four of these will be done by direct calculation from the definitions.
	%This means $a = A\cos(\theta)$, $b = A\sin(\theta)$, $c = C\cos(\phi)$, and $d = C\sin(\phi)$.
		\begin{enumerate}[(a)]
			\item $\conj{(w+z)} = \conj{a+ib+c+id} = \conj{a+c+i(b+d)} = a+c-i(b+d) = a-ib+c-id = \conj{a+ib}+\conj{c+id} = \conj{w}+\conj{z}$
			
			\item $\conj{w \cdot z} = \conj{(a=ib)(c+id)} = \conj{ac-bd+i(ad+bc)} = ac-bd-i(ad+bc) = (a-ib)(c-id) = (\conj{a+ib})(\conj{c+id}) = \conj{w} \cdot \conj{z}$
			
			\item $|wz| = |Ae^{i\theta}Ce^{i\phi}| = |ACe^{i(\theta+\phi)}| = AC = |Ae^{i\theta}||Ce^{i\phi}| = |w||z|$
			
			\item $z\conj{z} = (c+id)(c-id) = c^2+d^2 = \sqrt{c^2+d^2}^2 = |c+id|^2 = |z|^2$
			
			\item This inequality is an equality for $n=1$.
			Since addition of complex numbers is associative, we can always choose to group the addition into two groups.
			Therefore, if we show that this inequality holds for $n=2$, we have shown that it holds for all $n$.
			\begin{equation*} \begin{split}
				|z_1+z_2|^2 & = (z_1+z_2)\conj{(z_1+z_2)} \\
							& = (z_1+z_2)(\conj{z_1}+\conj{z_2}) \\
							& = z_1\conj{z_1} + z_2\conj{z_2}+z_1\conj{z_2}+z_2\conj{z_1} \\
							& \leq z_1\conj{z_1} + z_2\conj{z_2} + 2\sqrt{z_1\conj{z_1}z_2\conj{z_2}} \\
							& = (|z_1|+|z_2|)^2
			\end{split} \end{equation*}
			Thus $|z_1+z_2| \leq |z_1|+|z_2|$
			
			\item Since $|z|^2 = x^2+y^2$, we can conclude that $|x| \leq |z|$ and $|y| \leq |z|$. Let $m = \max(|x|,|y|)$. Then $m \leq |z|$ (because both $|x|$ and $|y|$ are) and
			$$ |z|^2 = |x|^2+|y|^2 \leq m^2+m^2 = 2m^2 \Rightarrow |z| \leq \sqrt{2}m $$
			Therefore, $m \leq |z| \leq \sqrt{2}m$.
			
			\item This can be done by direct calculation.
			$$ \frac{z+\conj{z}}{2} = \frac{a+ib+a-ib}{2} = \frac{2a}{2} = a = \Re(z) $$
			$$ \frac{z-\conj{z}}{2i} = \frac{a+ib-a-ib}{2i} = \frac{2ib}{2i} = b = \Im(z) $$	
		\end{enumerate}
	
	\item [A.3] 
\end{itemize}

\end{document}