\documentclass[../../Solutions.tex]{subfiles}

\begin{document}

\begin{itemize}
	\item [2.1.1] Let $E \subseteq (M,d)$ a metric space, and define $E^C = M \setminus E$. \\
		$(\Rightarrow)$ Assume $E$ is open.
		Let $x$ be a limit point of $E^C$, then either $x \in E$ or $x \in E^C$.
		If $x \in E$, then there exists $r > 0$ such that $B_r(x) \subseteq E$ since $E$ is open.
		But $B_r(x) \cap E^C = \emptyset$.
		Thus $x \in E^C$.
		Therefore $E^C$ contains all of its limit points and therefore is closed. \\
		$(\Leftarrow)$ Assume $E^C$ is closed.
		Let $x \in E$.
		Since $E^C$ is closed, there must be an $r > 0$ such that $B_r(x) \cap E^C = \emptyset$.
		(Otherwise, if no such $r$ exists, then $x$ would be a limit point of $E^C$ that is not in $E^C$.)
		Thus $B_r(x) \subseteq \left(E^C\right)^C = E$.
		Therefore $E$ is open.
	
	\item [2.1.2] We will use the equivalence given in Exercise 2.1.1 for parts (b) and (d).
	\begin{enumerate}[(a)]
		\item Assume $\{E_\alpha\}$ is a collection of open sets.
			Let $x \in \bigcup_\alpha E_\alpha$.
			Then $x \in E_\alpha$ for some $\alpha$.
			Thus there exists an $r > 0$ such that $B_r(x) \subseteq E_\alpha \subseteq \bigcup_\alpha E_\alpha$.
			Therefore $\bigcup_\alpha E_\alpha$ is open.
		\item Assume $\{F_\alpha\}$ is a collection of closed sets.
			Then $\{F_\alpha^C\}$ is a collection of open sets.
			Thus (by part (a)) $\bigcup_\alpha F_\alpha^C$ is open, which means its complement is closed.
			Thus $\left(\bigcup_\alpha F_\alpha^C\right)^C = \bigcap_\alpha F_\alpha$ is closed.
		\item Assume $\{E_i\}_{i=1}^N$ is a finite collection of open sets.
			Let $x \in \bigcap_{i=1}^N E_i$.
			Then $x \in E_i$ for all $i$, which means there exists $r_i$ such that $B_{r_i}(x) \subseteq E_i$ for all $i$.
			Set $r = \min{r_i}$.
			Then $B_r(x) \subseteq B_{r_i}(x) \subseteq E_i$ for all $i$.
			Thus $B_r(x) \subseteq \bigcap_{i=1}^N E_i$.
			Therefore $\bigcap_{i=1}^N E_i$ is open.
		\item Assume $\{F_i\}_{i=1}^N$ is a finite collection of closed sets.
			Then $\{F_i^C\}$ is a finite collection of open sets.
			Thus (by part (c)), $\bigcap_{i=1}^N F_i^C$ is open, which means its complement is closed.
			Therefore $\left(\bigcap_{i=1}^N F_i^C\right)^C = \bigcup_{i=1}^N F_i$ is closed.
	\end{enumerate}
	
	\item [2.1.3] The collection $\{(-1,1/n) : n \in \mathbb{N} \}$ is a collection of open sets, but
		$$ \bigcup_{n=1}^\infty \left(-1,\frac{1}{n}\right) = (-1,0] $$
		which is not an open set. \\
		The collection $\{[0,1-1/n] : n \in \mathbb{N} \}$ is a collection of closed sets, but
		$$ \bigcup_{n=1}^\infty \left[0,1-\frac{1}{n}\right] = [0,1) $$
		which is not a closed set.
	
	\item [2.1.4] Let $(M,d)$ be a metric space and let $E \subseteq M$.
		\textit{Note:} We use the claim that $E$ is closed if and only if $E^C$ is open --- this is proven in Exercise 2.1.1. \\
		Assume $E$ is compact and let $x \in E^C$.
		Define $r_e = d(e,x)/2$ for all $e \in E$.
		Define an open cover of $E$ as
		$$ \mathcal{F} = \left\{ B_{r_e}(e) : e \in E \right\} $$
		Since $E$ is compact, $\mathcal{F}$ has a finite sub-cover
		$$ \left\{ B_{r_{e_1}}(e_1), \dots , B_{r_{e_N}}(e_N) \right\} $$
		Let $\rho = \min{r_{e_1}, \dots , r_{e_N}}$.
		Suppose $y \in B_\rho(x) \cap E$. Then $y \in B_{r_{e_i}}(x)$ for some $i$ in the finite cover.
		Then
		$$ d(x,e_i) \leq d(x,y)+d(y,e_i) < \rho + r_{e_i} \leq 2r_{e_i} = d(x,e_i) $$
		which is impossible.
		So there is no $y \in E$ that is in $B_\rho(x)$.
		Thus $B_\rho(x) \subseteq E^C$, which means $E^C$ is open.
		Therefore $E$ is closed.
	
	\item [2.1.5]
	\begin{enumerate}[(a)]
		\item In this space $(0,1)$ is open and not closed.
			\textit{Proof:} Let $x \in (0,1)$.
			Then set $r = \min{x,1-x}$.
			Then $B_r(x) \subseteq (0,1)$.
			Thus $(0,1)$ is open. $0$ is a limit point of $(0,1)$ and $0 \not\in (0,1)$, so $(0,1)$ is not closed.
		\item In this space $(0,1)$ is neither open nor closed.
			\textit{Proof:} The point $(1/2,1) \in (0,1)$ but any ball around it will include points off the x-axis, so $(0,1)$ cannot be open.
			The point $(0,0)$ is a limit point of $(0,1)$ that is not in it, so $(0,1)$ cannot be closed.
	\end{enumerate}
	
	\item [2.1.6] Generally $\interior{A\cup B} \neq \interior{A} \cup \interior{B}$.
		Counter-example: Let $A = (0,1]$ and $B = [1,2)$ in $\R$.
		Then $\interior{A\cup B} = (0,2)$ and $\interior{A}\cup\interior{B} = (0,1)\cup(1,2) \neq (0,2)$. \\
		Contrarily, it is generally true that $\interior{A\cap B} = \interior{A} \cap\interior{B}$. \\
		\textit{Proof:} $(\subseteq)$ Let $x \in \interior{A\cap B}$.
		Then there exists $r>0$ such that $B_r(x) \subseteq A\cap B$.
		This means $B_r(x) \subseteq A$ and $B_r(x) \subseteq B$.
		Thus $x$ is in both interiors, meaning $x \in \interior{A}\cap\interior{B}$. \\
		$(\supseteq)$ Let $x \in \interior{A}\cap\interior{B}$.
		Then there exist $r_1,r_2$ such that $B_{r_1}(x) \subseteq A$ and $B_{r_2}(x) \subseteq B$.
		Set $r = \min{r_1,r_2}$.
		Then $B_r(x) \subseteq A\cap B$.
		Thus $x \in \interior{A\cap B}$. \\
	
	
\end{itemize}

\end{document}