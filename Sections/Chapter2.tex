\documentclass[../../Solutions.tex]{subfiles}

\begin{document}

\begin{itemize}
	\item [2.1.1] Let $E \subseteq (M,d)$ a metric space, and define $E^C = M \setminus E$. \\
		$(\Rightarrow)$ Assume $E$ is open.
		Let $x$ be a limit point of $E^C$, then either $x \in E$ or $x \in E^C$.
		If $x \in E$, then there exists $r > 0$ such that $B_r(x) \subseteq E$ since $E$ is open.
		But $B_r(x) \cap E^C = \emptyset$.
		Thus $x \in E^C$.
		Therefore $E^C$ contains all of its limit points and therefore is closed. \\
		$(\Leftarrow)$ Assume $E^C$ is closed.
		Let $x \in E$.
		Since $E^C$ is closed, there must be an $r > 0$ such that $B_r(x) \cap E^C = \emptyset$.
		(Otherwise, if no such $r$ exists, then $x$ would be a limit point of $E^C$ that is not in $E^C$.)
		Thus $B_r(x) \subseteq \left(E^C\right)^C = E$.
		Therefore $E$ is open.
	
	\item [2.1.2] We will use the equivalence given in Exercise 2.1.1 for parts (b) and (d).
	\begin{enumerate}[(a)]
		\item Assume $\{E_\alpha\}$ is a collection of open sets.
			Let $x \in \cup_\alpha E_\alpha$.
			Then $x \in E_\alpha$ for some $\alpha$.
			Thus there exists an $r > 0$ such that $B_r(x) \subseteq E_\alpha \subseteq \cup_\alpha E_\alpha$.
			Therefore $\cup_\alpha E_\alpha$ is open.
		\item Assume $\{F_\alpha\}$ is a collection of closed sets.
			Then $\{F_\alpha^C\}$ is a collection of open sets.
			Thus (by part (a)) $\cup_\alpha F_\alpha^C$ is open, which means its complement is closed.
			Thus $\left(\cup_\alpha F_\alpha^C\right)^C = \cap_\alpha F_\alpha$ is closed.
		\item Assume $\{E_i\}_{i=1}^N$ is a finite collection of open sets.
			Let 
	\end{enumerate}
\end{itemize}

\end{document}