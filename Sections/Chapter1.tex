\documentclass[../../Solutions.tex]{subfiles}

\begin{document}

\begin{itemize}
	\item [1.1.1]
		\begin{enumerate}[(a)]
			\item We will verify each norm by showing that it satisfies each of the required conditions using the numbering scheme in the text. These verifications are mostly simple algebra and small logical steps.
				Let $u,v \in \R^2$ and $\lambda \in \R$. \\
				Let $\norm{\cdot}_1:\R^2 \to \R$ be defined as $\norm{u}_1 = |u_1|+|u_2|$.
				\begin{enumerate}[(i)]
					\item $\norm{u}_1 = |u_1|+|u_2| \geq 0$ because the sum of two positive numbers is always positive.
					\item $\norm{u}_1 = 0 \iff |u_1|+|u_2| = 0 \iff u_1 = 0 \text{ and } u_2 = 0 \iff u = 0$
					\item $\norm{\lambda u}_1 = |\lambda u_1|+|\lambda u_2| = |\lambda|(|u_1|+|u_2|) = |\lambda|\norm{u}_1$
					\item $\norm{u+v}_1 = |u_1+v_1|+|u_2+v_2| \leq |u_1|+|v_1|+|u_2|+|v_2| = \norm{u}_1+\norm{v}_1$
				\end{enumerate}
				Let $\norm{\cdot}_2:\R^2 \to \R$ be defined as $\norm{u}_2 = \sqrt{u_1^2+u_2^2}$.
				\begin{enumerate}[(i)]
					\item $\norm{u}_2 = \sqrt{u_1^2+u_2^2} \geq 0$ because the sum of two positive numbers is always positive.
					\item $\norm{u}_2 = 0 \iff \sqrt{u_1^2+u_2^2} = 0 \iff u_1^2+u_2^2 = 0 \iff u_1 = 0 \text{ and } u_2 = 0 \iff u = 0$
					\item $\norm{\lambda u}_2 = \sqrt{(\lambda u_1)^2+(\lambda u_2)^2} = \sqrt{\lambda^2(u_1^2+u_2^2)} = |\lambda|\sqrt{u_1^2+u_2^2} = |\lambda|\norm{u}_2$
					\item $\norm{u+v}_2 = \sqrt{(u_1+v_1)^2+(u_2+v_2)^2} = \sqrt{u_1^2+v_1^2+2u_1v_1+u_2^2+v_2^2+2u_2v_2} \leq \sqrt{u_1^2+u_2^2}+\sqrt{v_1^2+v_2^2} = \norm{u}_2+\norm{v}_2$
				\end{enumerate}
				Let $\norm{\cdot}_\infty:\R^2 \to \R$ be defined as $\norm{u}_\infty = \max(|u_1|,|u_2|)$.
				\begin{enumerate}[(i)]
					\item $\norm{u}_\infty \geq 0$ because both $|u_1| \geq 0$ and $|u_2| \geq 0$.
					\item $\norm{u}_\infty = 0 \iff u_1 = 0 \text{ and } u_2 = 0 \iff u = 0$
					\item $\norm{\lambda u}_\infty = \max(|\lambda u_1|,|\lambda u_2|) = |\lambda|\max(|u_1|,|u_2|) = |\lambda|\norm{u}_\infty$
					\item $\norm{u+v}_\infty = \max(|u_1+v_1|,|u_2+v_2|) \leq \max(|u_1|+|v_1|,|u_2|+|v_2|) \leq \max(|u_1|,|u_2|)+\max(|v_1|,|v_2|) = \norm{u}_\infty+\norm{v}_\infty$
				\end{enumerate}
			\item We obtain a metric $d$ from a norm $\norm{\cdot}$ by $d(u,v) = \norm{u-v}$, so
				$$ d_1((1,1),(2,3)) = \norm{(1,1)-(2,3)}_1 = |1-2|+|1-3| = 1+2 = 3 $$
				$$ d_2((1,1),(2,3)) = \norm{(1,1)-(2,3)}_2 = \sqrt{(1-2)^2+(1-3)^2} = \sqrt{1+4} = \sqrt{5} $$
				$$ d_\infty((1,1),(2,3)) = \norm{(1,1)-(2,3)}_\infty = \max(|1-2|,|1-3|) = \max(1,2) = 2 $$
			\item Below are the example $r$-balls for the different norms.
			\begin{figure}[htp] %PLACE CORRECTLY
				\begin{center}
					% Graphic for TeX using PGF
% Title: /home/tom/Documents/BeginningFunctionalAnalysis/figures/1_1_1_c.dia
% Creator: Dia v0.97.3
% CreationDate: Wed Feb 14 20:54:28 2018
% For: tom
% \usepackage{tikz}
% The following commands are not supported in PSTricks at present
% We define them conditionally, so when they are implemented,
% this pgf file will use them.
\ifx\du\undefined
  \newlength{\du}
\fi
\setlength{\du}{15\unitlength}
\begin{tikzpicture}
\pgftransformxscale{1.000000}
\pgftransformyscale{-1.000000}
\definecolor{dialinecolor}{rgb}{0.000000, 0.000000, 0.000000}
\pgfsetstrokecolor{dialinecolor}
\definecolor{dialinecolor}{rgb}{1.000000, 1.000000, 1.000000}
\pgfsetfillcolor{dialinecolor}
\pgfsetlinewidth{0.150000\du}
\pgfsetdash{}{0pt}
\pgfsetdash{}{0pt}
\pgfsetbuttcap
\pgfsetmiterjoin
\pgfsetlinewidth{0.150000\du}
\pgfsetbuttcap
\pgfsetmiterjoin
\pgfsetdash{}{0pt}
\definecolor{dialinecolor}{rgb}{0.000000, 0.000000, 0.000000}
\pgfsetstrokecolor{dialinecolor}
\pgfpathellipse{\pgfpoint{10.000000\du}{0.000000\du}}{\pgfpoint{1.000000\du}{0\du}}{\pgfpoint{0\du}{1.000000\du}}
\pgfusepath{stroke}
\pgfsetbuttcap
\pgfsetmiterjoin
\pgfsetdash{}{0pt}
\definecolor{dialinecolor}{rgb}{0.000000, 0.000000, 0.000000}
\pgfsetstrokecolor{dialinecolor}
\pgfpathellipse{\pgfpoint{10.000000\du}{0.000000\du}}{\pgfpoint{1.000000\du}{0\du}}{\pgfpoint{0\du}{1.000000\du}}
\pgfusepath{stroke}
\pgfsetlinewidth{0.150000\du}
\pgfsetdash{}{0pt}
\pgfsetdash{}{0pt}
\pgfsetbuttcap
\pgfsetmiterjoin
\pgfsetlinewidth{0.150000\du}
\pgfsetbuttcap
\pgfsetmiterjoin
\pgfsetdash{}{0pt}
\definecolor{dialinecolor}{rgb}{0.000000, 0.000000, 0.000000}
\pgfsetstrokecolor{dialinecolor}
\pgfpathellipse{\pgfpoint{12.000000\du}{-2.000000\du}}{\pgfpoint{3.000000\du}{0\du}}{\pgfpoint{0\du}{3.000000\du}}
\pgfusepath{stroke}
\pgfsetbuttcap
\pgfsetmiterjoin
\pgfsetdash{}{0pt}
\definecolor{dialinecolor}{rgb}{0.000000, 0.000000, 0.000000}
\pgfsetstrokecolor{dialinecolor}
\pgfpathellipse{\pgfpoint{12.000000\du}{-2.000000\du}}{\pgfpoint{3.000000\du}{0\du}}{\pgfpoint{0\du}{3.000000\du}}
\pgfusepath{stroke}
\pgfsetlinewidth{0.150000\du}
\pgfsetdash{}{0pt}
\pgfsetdash{}{0pt}
\pgfsetbuttcap
\pgfsetmiterjoin
\pgfsetlinewidth{0.150000\du}
\pgfsetbuttcap
\pgfsetmiterjoin
\pgfsetdash{}{0pt}
\definecolor{dialinecolor}{rgb}{0.000000, 0.000000, 0.000000}
\pgfsetstrokecolor{dialinecolor}
\draw (2.000000\du,-5.000000\du)--(5.000000\du,-2.000000\du)--(2.000000\du,1.000000\du)--(-1.000000\du,-2.000000\du)--cycle;
\pgfsetbuttcap
\pgfsetmiterjoin
\pgfsetdash{}{0pt}
\definecolor{dialinecolor}{rgb}{0.000000, 0.000000, 0.000000}
\pgfsetstrokecolor{dialinecolor}
\draw (2.000000\du,-5.000000\du)--(5.000000\du,-2.000000\du)--(2.000000\du,1.000000\du)--(-1.000000\du,-2.000000\du)--cycle;
\pgfsetlinewidth{0.150000\du}
\pgfsetdash{}{0pt}
\pgfsetdash{}{0pt}
\pgfsetbuttcap
\pgfsetmiterjoin
\pgfsetlinewidth{0.150000\du}
\pgfsetbuttcap
\pgfsetmiterjoin
\pgfsetdash{}{0pt}
\definecolor{dialinecolor}{rgb}{0.000000, 0.000000, 0.000000}
\pgfsetstrokecolor{dialinecolor}
\draw (0.000000\du,-1.000000\du)--(1.000000\du,0.000000\du)--(0.000000\du,1.000000\du)--(-1.000000\du,0.000000\du)--cycle;
\pgfsetbuttcap
\pgfsetmiterjoin
\pgfsetdash{}{0pt}
\definecolor{dialinecolor}{rgb}{0.000000, 0.000000, 0.000000}
\pgfsetstrokecolor{dialinecolor}
\draw (0.000000\du,-1.000000\du)--(1.000000\du,0.000000\du)--(0.000000\du,1.000000\du)--(-1.000000\du,0.000000\du)--cycle;
\pgfsetlinewidth{0.150000\du}
\pgfsetdash{}{0pt}
\pgfsetdash{}{0pt}
\pgfsetbuttcap
\pgfsetmiterjoin
\pgfsetlinewidth{0.150000\du}
\pgfsetbuttcap
\pgfsetmiterjoin
\pgfsetdash{}{0pt}
\definecolor{dialinecolor}{rgb}{1.000000, 1.000000, 1.000000}
\pgfsetfillcolor{dialinecolor}
\pgfpathellipse{\pgfpoint{-0.025000\du}{0.025000\du}}{\pgfpoint{0.075000\du}{0\du}}{\pgfpoint{0\du}{0.075000\du}}
\pgfusepath{fill}
\definecolor{dialinecolor}{rgb}{0.000000, 0.000000, 0.000000}
\pgfsetstrokecolor{dialinecolor}
\pgfpathellipse{\pgfpoint{-0.025000\du}{0.025000\du}}{\pgfpoint{0.075000\du}{0\du}}{\pgfpoint{0\du}{0.075000\du}}
\pgfusepath{stroke}
\pgfsetbuttcap
\pgfsetmiterjoin
\pgfsetdash{}{0pt}
\definecolor{dialinecolor}{rgb}{0.000000, 0.000000, 0.000000}
\pgfsetstrokecolor{dialinecolor}
\pgfpathellipse{\pgfpoint{-0.025000\du}{0.025000\du}}{\pgfpoint{0.075000\du}{0\du}}{\pgfpoint{0\du}{0.075000\du}}
\pgfusepath{stroke}
\pgfsetlinewidth{0.150000\du}
\pgfsetdash{}{0pt}
\pgfsetdash{}{0pt}
\pgfsetbuttcap
\pgfsetmiterjoin
\pgfsetlinewidth{0.150000\du}
\pgfsetbuttcap
\pgfsetmiterjoin
\pgfsetdash{}{0pt}
\definecolor{dialinecolor}{rgb}{1.000000, 1.000000, 1.000000}
\pgfsetfillcolor{dialinecolor}
\pgfpathellipse{\pgfpoint{1.975000\du}{-2.025000\du}}{\pgfpoint{0.075000\du}{0\du}}{\pgfpoint{0\du}{0.075000\du}}
\pgfusepath{fill}
\definecolor{dialinecolor}{rgb}{0.000000, 0.000000, 0.000000}
\pgfsetstrokecolor{dialinecolor}
\pgfpathellipse{\pgfpoint{1.975000\du}{-2.025000\du}}{\pgfpoint{0.075000\du}{0\du}}{\pgfpoint{0\du}{0.075000\du}}
\pgfusepath{stroke}
\pgfsetbuttcap
\pgfsetmiterjoin
\pgfsetdash{}{0pt}
\definecolor{dialinecolor}{rgb}{0.000000, 0.000000, 0.000000}
\pgfsetstrokecolor{dialinecolor}
\pgfpathellipse{\pgfpoint{1.975000\du}{-2.025000\du}}{\pgfpoint{0.075000\du}{0\du}}{\pgfpoint{0\du}{0.075000\du}}
\pgfusepath{stroke}
% setfont left to latex
\definecolor{dialinecolor}{rgb}{0.000000, 0.000000, 0.000000}
\pgfsetstrokecolor{dialinecolor}
\node[anchor=west] at (0.200000\du,-2.600000\du){$B_3(2,2)$};
% setfont left to latex
\definecolor{dialinecolor}{rgb}{0.000000, 0.000000, 0.000000}
\pgfsetstrokecolor{dialinecolor}
\node[anchor=west] at (-3.200000\du,-0.800000\du){$B_1(0)$};
\pgfsetlinewidth{0.150000\du}
\pgfsetdash{}{0pt}
\pgfsetdash{}{0pt}
\pgfsetbuttcap
\pgfsetmiterjoin
\pgfsetlinewidth{0.150000\du}
\pgfsetbuttcap
\pgfsetmiterjoin
\pgfsetdash{}{0pt}
\definecolor{dialinecolor}{rgb}{1.000000, 1.000000, 1.000000}
\pgfsetfillcolor{dialinecolor}
\pgfpathellipse{\pgfpoint{9.975000\du}{0.025000\du}}{\pgfpoint{0.075000\du}{0\du}}{\pgfpoint{0\du}{0.075000\du}}
\pgfusepath{fill}
\definecolor{dialinecolor}{rgb}{0.000000, 0.000000, 0.000000}
\pgfsetstrokecolor{dialinecolor}
\pgfpathellipse{\pgfpoint{9.975000\du}{0.025000\du}}{\pgfpoint{0.075000\du}{0\du}}{\pgfpoint{0\du}{0.075000\du}}
\pgfusepath{stroke}
\pgfsetbuttcap
\pgfsetmiterjoin
\pgfsetdash{}{0pt}
\definecolor{dialinecolor}{rgb}{0.000000, 0.000000, 0.000000}
\pgfsetstrokecolor{dialinecolor}
\pgfpathellipse{\pgfpoint{9.975000\du}{0.025000\du}}{\pgfpoint{0.075000\du}{0\du}}{\pgfpoint{0\du}{0.075000\du}}
\pgfusepath{stroke}
\pgfsetlinewidth{0.150000\du}
\pgfsetdash{}{0pt}
\pgfsetdash{}{0pt}
\pgfsetbuttcap
\pgfsetmiterjoin
\pgfsetlinewidth{0.150000\du}
\pgfsetbuttcap
\pgfsetmiterjoin
\pgfsetdash{}{0pt}
\definecolor{dialinecolor}{rgb}{1.000000, 1.000000, 1.000000}
\pgfsetfillcolor{dialinecolor}
\pgfpathellipse{\pgfpoint{11.975000\du}{-2.025000\du}}{\pgfpoint{0.075000\du}{0\du}}{\pgfpoint{0\du}{0.075000\du}}
\pgfusepath{fill}
\definecolor{dialinecolor}{rgb}{0.000000, 0.000000, 0.000000}
\pgfsetstrokecolor{dialinecolor}
\pgfpathellipse{\pgfpoint{11.975000\du}{-2.025000\du}}{\pgfpoint{0.075000\du}{0\du}}{\pgfpoint{0\du}{0.075000\du}}
\pgfusepath{stroke}
\pgfsetbuttcap
\pgfsetmiterjoin
\pgfsetdash{}{0pt}
\definecolor{dialinecolor}{rgb}{0.000000, 0.000000, 0.000000}
\pgfsetstrokecolor{dialinecolor}
\pgfpathellipse{\pgfpoint{11.975000\du}{-2.025000\du}}{\pgfpoint{0.075000\du}{0\du}}{\pgfpoint{0\du}{0.075000\du}}
\pgfusepath{stroke}
% setfont left to latex
\definecolor{dialinecolor}{rgb}{0.000000, 0.000000, 0.000000}
\pgfsetstrokecolor{dialinecolor}
\node[anchor=west] at (10.200000\du,-2.600000\du){$B_3(2,2)$};
% setfont left to latex
\definecolor{dialinecolor}{rgb}{0.000000, 0.000000, 0.000000}
\pgfsetstrokecolor{dialinecolor}
\node[anchor=west] at (6.000000\du,-0.500000\du){$B_1(0)$};
\pgfsetlinewidth{0.150000\du}
\pgfsetdash{}{0pt}
\pgfsetdash{}{0pt}
\pgfsetbuttcap
\pgfsetmiterjoin
\pgfsetlinewidth{0.150000\du}
\pgfsetbuttcap
\pgfsetmiterjoin
\pgfsetdash{}{0pt}
\definecolor{dialinecolor}{rgb}{1.000000, 1.000000, 1.000000}
\pgfsetfillcolor{dialinecolor}
\pgfpathellipse{\pgfpoint{19.975000\du}{0.025000\du}}{\pgfpoint{0.075000\du}{0\du}}{\pgfpoint{0\du}{0.075000\du}}
\pgfusepath{fill}
\definecolor{dialinecolor}{rgb}{0.000000, 0.000000, 0.000000}
\pgfsetstrokecolor{dialinecolor}
\pgfpathellipse{\pgfpoint{19.975000\du}{0.025000\du}}{\pgfpoint{0.075000\du}{0\du}}{\pgfpoint{0\du}{0.075000\du}}
\pgfusepath{stroke}
\pgfsetbuttcap
\pgfsetmiterjoin
\pgfsetdash{}{0pt}
\definecolor{dialinecolor}{rgb}{0.000000, 0.000000, 0.000000}
\pgfsetstrokecolor{dialinecolor}
\pgfpathellipse{\pgfpoint{19.975000\du}{0.025000\du}}{\pgfpoint{0.075000\du}{0\du}}{\pgfpoint{0\du}{0.075000\du}}
\pgfusepath{stroke}
\pgfsetlinewidth{0.150000\du}
\pgfsetdash{}{0pt}
\pgfsetdash{}{0pt}
\pgfsetbuttcap
\pgfsetmiterjoin
\pgfsetlinewidth{0.150000\du}
\pgfsetbuttcap
\pgfsetmiterjoin
\pgfsetdash{}{0pt}
\definecolor{dialinecolor}{rgb}{1.000000, 1.000000, 1.000000}
\pgfsetfillcolor{dialinecolor}
\pgfpathellipse{\pgfpoint{21.975000\du}{-2.025000\du}}{\pgfpoint{0.075000\du}{0\du}}{\pgfpoint{0\du}{0.075000\du}}
\pgfusepath{fill}
\definecolor{dialinecolor}{rgb}{0.000000, 0.000000, 0.000000}
\pgfsetstrokecolor{dialinecolor}
\pgfpathellipse{\pgfpoint{21.975000\du}{-2.025000\du}}{\pgfpoint{0.075000\du}{0\du}}{\pgfpoint{0\du}{0.075000\du}}
\pgfusepath{stroke}
\pgfsetbuttcap
\pgfsetmiterjoin
\pgfsetdash{}{0pt}
\definecolor{dialinecolor}{rgb}{0.000000, 0.000000, 0.000000}
\pgfsetstrokecolor{dialinecolor}
\pgfpathellipse{\pgfpoint{21.975000\du}{-2.025000\du}}{\pgfpoint{0.075000\du}{0\du}}{\pgfpoint{0\du}{0.075000\du}}
\pgfusepath{stroke}
% setfont left to latex
\definecolor{dialinecolor}{rgb}{0.000000, 0.000000, 0.000000}
\pgfsetstrokecolor{dialinecolor}
\node[anchor=west] at (20.200000\du,-2.600000\du){$B_3(2,2)$};
% setfont left to latex
\definecolor{dialinecolor}{rgb}{0.000000, 0.000000, 0.000000}
\pgfsetstrokecolor{dialinecolor}
\node[anchor=west] at (16.000000\du,-0.500000\du){$B_1(0)$};
\pgfsetlinewidth{0.150000\du}
\pgfsetdash{}{0pt}
\pgfsetdash{}{0pt}
\pgfsetbuttcap
\pgfsetmiterjoin
\pgfsetlinewidth{0.150000\du}
\pgfsetbuttcap
\pgfsetmiterjoin
\pgfsetdash{}{0pt}
\definecolor{dialinecolor}{rgb}{0.000000, 0.000000, 0.000000}
\pgfsetstrokecolor{dialinecolor}
\draw (19.032258\du,-1.000000\du)--(19.032258\du,1.000000\du)--(20.967742\du,1.000000\du)--(20.967742\du,-1.000000\du)--cycle;
\pgfsetbuttcap
\pgfsetmiterjoin
\pgfsetdash{}{0pt}
\definecolor{dialinecolor}{rgb}{0.000000, 0.000000, 0.000000}
\pgfsetstrokecolor{dialinecolor}
\draw (19.032258\du,-1.000000\du)--(19.032258\du,1.000000\du)--(20.967742\du,1.000000\du)--(20.967742\du,-1.000000\du)--cycle;
\pgfsetlinewidth{0.150000\du}
\pgfsetdash{}{0pt}
\pgfsetdash{}{0pt}
\pgfsetbuttcap
\pgfsetmiterjoin
\pgfsetlinewidth{0.150000\du}
\pgfsetbuttcap
\pgfsetmiterjoin
\pgfsetdash{}{0pt}
\definecolor{dialinecolor}{rgb}{0.000000, 0.000000, 0.000000}
\pgfsetstrokecolor{dialinecolor}
\draw (19.032258\du,-5.000000\du)--(19.032258\du,1.083333\du)--(24.919355\du,1.083333\du)--(24.919355\du,-5.000000\du)--cycle;
\pgfsetbuttcap
\pgfsetmiterjoin
\pgfsetdash{}{0pt}
\definecolor{dialinecolor}{rgb}{0.000000, 0.000000, 0.000000}
\pgfsetstrokecolor{dialinecolor}
\draw (19.032258\du,-5.000000\du)--(19.032258\du,1.083333\du)--(24.919355\du,1.083333\du)--(24.919355\du,-5.000000\du)--cycle;
% setfont left to latex
\definecolor{dialinecolor}{rgb}{0.000000, 0.000000, 0.000000}
\pgfsetstrokecolor{dialinecolor}
\node[anchor=west] at (-3.000000\du,-5.000000\du){$\norm{\cdot}_1$};
% setfont left to latex
\definecolor{dialinecolor}{rgb}{0.000000, 0.000000, 0.000000}
\pgfsetstrokecolor{dialinecolor}
\node[anchor=west] at (5.000000\du,-5.000000\du){$\norm{\cdot}_2$};
% setfont left to latex
\definecolor{dialinecolor}{rgb}{0.000000, 0.000000, 0.000000}
\pgfsetstrokecolor{dialinecolor}
\node[anchor=west] at (14.000000\du,-5.000000\du){$\norm{\cdot}_2$};
\end{tikzpicture}

				\end{center}
			\end{figure}
		\end{enumerate}
	
	\item [1.1.2]
		\begin{enumerate}[(a)]
			\item Let $\norm{\cdot}:\R^n \to \R$ be defined as $\norm{v} = (1/3)\norm{v}_1+(2/3)\norm{v}_\infty$.
				Let $x,y \in \R^n$ and $\lambda \in \R$.
				In each case, we will use the facts that $\norm{\cdot}_1$ and $\norm{\cdot}_\infty$ are norms.
				
				\begin{enumerate}[(i)]
					\item $\norm{x} = (1/3)\norm{x}_1+(2/3)\norm{x}_\infty \geq 0$
					\item $\norm{x} = 0 \iff (1/3)\norm{x}_1+(2/3)\norm{x}_\infty = 0 \iff \norm{x}_1 = 0 \text{ and } \norm{x}_\infty = 0 \iff x = 0$.
					\item \begin{equation*} \begin{split}
						 \norm{\lambda x} & = (1/3)\norm{\lambda x}_1+(2/3)\norm{\lambda x}_\infty = (1/3)|\lambda|\norm{x}_1+(2/3)|\lambda|\norm{x}_\infty \\
							 & = |\lambda|((1/3)\norm{x}_1+(2/3)\norm{x}_\infty) = |\lambda|\norm{x}
					\end{split} \end{equation*}
					\item $\norm{x+y} = (1/3)\norm{x+y}_1+(2/3)\norm{x+y}_\infty \leq (1/3)(\norm{x}_1+\norm{y}_1)+(2/3)(\norm{x}_\infty+\norm{y}_\infty) = \norm{x}+\norm{y}$.
				\end{enumerate}
				Thus $\norm{\cdot}$ satisfies the conditions to be a norm on $\R^n$.
			\item Below is the unit ball around the origin for $\norm{\cdot}$.
			\begin{figure}[h]
				\begin{center}
					% Graphic for TeX using PGF
% Title: /home/tom/Documents/BeginningFunctionalAnalysis/figures/1_1_2_b.dia
% Creator: Dia v0.97.3
% CreationDate: Fri Feb 16 22:11:45 2018
% For: tom
% \usepackage{tikz}
% The following commands are not supported in PSTricks at present
% We define them conditionally, so when they are implemented,
% this pgf file will use them.
\ifx\du\undefined
  \newlength{\du}
\fi
\setlength{\du}{15\unitlength}
\begin{tikzpicture}
\pgftransformxscale{1.000000}
\pgftransformyscale{-1.000000}
\definecolor{dialinecolor}{rgb}{0.000000, 0.000000, 0.000000}
\pgfsetstrokecolor{dialinecolor}
\definecolor{dialinecolor}{rgb}{1.000000, 1.000000, 1.000000}
\pgfsetfillcolor{dialinecolor}
\pgfsetlinewidth{0.150000\du}
\pgfsetdash{}{0pt}
\pgfsetdash{}{0pt}
\pgfsetbuttcap
\pgfsetmiterjoin
\pgfsetlinewidth{0.150000\du}
\pgfsetbuttcap
\pgfsetmiterjoin
\pgfsetdash{}{0pt}
\definecolor{dialinecolor}{rgb}{1.000000, 1.000000, 1.000000}
\pgfsetfillcolor{dialinecolor}
\pgfpathellipse{\pgfpoint{18.986047\du}{17.025649\du}}{\pgfpoint{0.025394\du}{0\du}}{\pgfpoint{0\du}{0.025394\du}}
\pgfusepath{fill}
\definecolor{dialinecolor}{rgb}{0.000000, 0.000000, 0.000000}
\pgfsetstrokecolor{dialinecolor}
\pgfpathellipse{\pgfpoint{18.986047\du}{17.025649\du}}{\pgfpoint{0.025394\du}{0\du}}{\pgfpoint{0\du}{0.025394\du}}
\pgfusepath{stroke}
\pgfsetbuttcap
\pgfsetmiterjoin
\pgfsetdash{}{0pt}
\definecolor{dialinecolor}{rgb}{0.000000, 0.000000, 0.000000}
\pgfsetstrokecolor{dialinecolor}
\pgfpathellipse{\pgfpoint{18.986047\du}{17.025649\du}}{\pgfpoint{0.025394\du}{0\du}}{\pgfpoint{0\du}{0.025394\du}}
\pgfusepath{stroke}
\pgfsetlinewidth{0.100000\du}
\pgfsetdash{}{0pt}
\pgfsetdash{}{0pt}
\pgfsetbuttcap
{
\definecolor{dialinecolor}{rgb}{0.000000, 0.000000, 0.000000}
\pgfsetfillcolor{dialinecolor}
% was here!!!
\definecolor{dialinecolor}{rgb}{0.000000, 0.000000, 0.000000}
\pgfsetstrokecolor{dialinecolor}
\draw (19.700000\du,16.300000\du)--(20.000000\du,17.000000\du);
}
\pgfsetlinewidth{0.100000\du}
\pgfsetdash{}{0pt}
\pgfsetdash{}{0pt}
\pgfsetbuttcap
{
\definecolor{dialinecolor}{rgb}{0.000000, 0.000000, 0.000000}
\pgfsetfillcolor{dialinecolor}
% was here!!!
\definecolor{dialinecolor}{rgb}{0.000000, 0.000000, 0.000000}
\pgfsetstrokecolor{dialinecolor}
\draw (19.000000\du,16.000000\du)--(19.700000\du,16.300000\du);
}
\pgfsetlinewidth{0.100000\du}
\pgfsetdash{}{0pt}
\pgfsetdash{}{0pt}
\pgfsetbuttcap
{
\definecolor{dialinecolor}{rgb}{0.000000, 0.000000, 0.000000}
\pgfsetfillcolor{dialinecolor}
% was here!!!
\definecolor{dialinecolor}{rgb}{0.000000, 0.000000, 0.000000}
\pgfsetstrokecolor{dialinecolor}
\draw (19.700000\du,17.700000\du)--(20.000000\du,17.000000\du);
}
\pgfsetlinewidth{0.100000\du}
\pgfsetdash{}{0pt}
\pgfsetdash{}{0pt}
\pgfsetbuttcap
{
\definecolor{dialinecolor}{rgb}{0.000000, 0.000000, 0.000000}
\pgfsetfillcolor{dialinecolor}
% was here!!!
\definecolor{dialinecolor}{rgb}{0.000000, 0.000000, 0.000000}
\pgfsetstrokecolor{dialinecolor}
\draw (19.000000\du,18.000000\du)--(19.700000\du,17.700000\du);
}
\pgfsetlinewidth{0.100000\du}
\pgfsetdash{}{0pt}
\pgfsetdash{}{0pt}
\pgfsetbuttcap
{
\definecolor{dialinecolor}{rgb}{0.000000, 0.000000, 0.000000}
\pgfsetfillcolor{dialinecolor}
% was here!!!
\definecolor{dialinecolor}{rgb}{0.000000, 0.000000, 0.000000}
\pgfsetstrokecolor{dialinecolor}
\draw (18.300000\du,17.700000\du)--(19.000000\du,18.000000\du);
}
\pgfsetlinewidth{0.100000\du}
\pgfsetdash{}{0pt}
\pgfsetdash{}{0pt}
\pgfsetbuttcap
{
\definecolor{dialinecolor}{rgb}{0.000000, 0.000000, 0.000000}
\pgfsetfillcolor{dialinecolor}
% was here!!!
\definecolor{dialinecolor}{rgb}{0.000000, 0.000000, 0.000000}
\pgfsetstrokecolor{dialinecolor}
\draw (18.000000\du,17.000000\du)--(18.300000\du,17.700000\du);
}
\pgfsetlinewidth{0.100000\du}
\pgfsetdash{}{0pt}
\pgfsetdash{}{0pt}
\pgfsetbuttcap
{
\definecolor{dialinecolor}{rgb}{0.000000, 0.000000, 0.000000}
\pgfsetfillcolor{dialinecolor}
% was here!!!
\definecolor{dialinecolor}{rgb}{0.000000, 0.000000, 0.000000}
\pgfsetstrokecolor{dialinecolor}
\draw (18.300000\du,16.300000\du)--(18.000000\du,17.000000\du);
}
\pgfsetlinewidth{0.100000\du}
\pgfsetdash{}{0pt}
\pgfsetdash{}{0pt}
\pgfsetbuttcap
{
\definecolor{dialinecolor}{rgb}{0.000000, 0.000000, 0.000000}
\pgfsetfillcolor{dialinecolor}
% was here!!!
\definecolor{dialinecolor}{rgb}{0.000000, 0.000000, 0.000000}
\pgfsetstrokecolor{dialinecolor}
\draw (19.000000\du,16.000000\du)--(18.300000\du,16.300000\du);
}
\end{tikzpicture}

				\end{center}
			\end{figure}
		\end{enumerate}
	
	\item [1.1.3] $\ell^1$ and $\ell^\infty$ are linear spaces using element-wise addition as the binary operation and the zero sequence $\mathbf{0} = \{0\}_{j=1}^\infty$ as the zero element.
	Now we must show that $\norm{\cdot}_1$ and $\norm{\cdot}_\infty$ are norms in their respective spaces.
	Let $\mathbf{a},\mathbf{b}$ be elements in the linear space and $\lambda \in \R$. \\
	First, $\norm{\cdot}_1$ satisfies the conditions to be a norm:
	\begin{enumerate}[(i)]
		\item $\norm{\mathbf{a}}_1 = \sum | a_i | \geq 0$ because each element of the sum $|a_i| \geq 0$.
		\item $\norm{\mathbf{a}}_1 = \sum | a_i | = 0 \iff |a_i| = 0 \text{ for all } i \iff \mathbf{a} = \mathbf{0}$.
		\item $\norm{\lambda\mathbf{a}}_1 = \sum | \lambda a_i | = \sum |\lambda||a_i| = |\lambda|\sum |a_i| = |\lambda|\norm{\mathbf{a}}_1$.
		\item This condition is satisfied by a central theorem in real analysis, which extends the triangle inequality to absolutely convergent series.
	\end{enumerate}
	
	Next, $\norm{\cdot}_\infty$ satisfies the conditions to be a norm:
	\begin{enumerate}[(i)]
		\item $\norm{\mathbf{a}}_\infty = \sup\{|a_i|\} \geq 0$ because each $|a_i| \geq 0$.
		\item $\norm{\mathbf{a}}_\infty = \sup\{|a_i|\} = 0 \iff |a_i| = 0 \text{ for all } i \iff \mathbf{a} = \mathbf{0}$.
		\item $\norm{\lambda\mathbf{a}}_\infty = \sup\{|\lambda a_i|\} = |\lambda|\sup\{|a_i|\} = |\lambda|\norm{\mathbf{a}}_\infty$.
		\item $\norm{\mathbf{a}+\mathbf{b}}_\infty = \sup\{|a_i+b_i|\} \leq \sup\{|a_i|+|b_i|\} \leq \sup\{|a_i|\}+\sup\{|b_i|\} = \norm{\mathbf{a}}_\infty + \norm{\mathbf{b}}_\infty$.
	\end{enumerate}
	Therefore $\ell^1$ and $\ell^\infty$ are both normed linear spaces.
	
	\item [1.1.4] $C([a,b])$ is a linear space with point-wise addition as the binary operation and $o(x)=0$ for all $x \in [a,b]$ as the zero element.
	All that is left now is showing that $\norm{\cdot}_\infty$ is a norm.
	Let $f,g \in C([a,b])$ and $\lambda \in \R$.
	\begin{enumerate}[(i)]
		\item $\norm{f}_\infty = \sup\{|f(x)|:x\in[a,b]\} \geq 0$ because $|f(x)| \geq 0$ for all $x \in [a,b]$.
		\item $\norm{f}_\infty = \sup\{|f(x)|:x\in[a,b]\} = 0 \iff |f(x)| = 0 \text{ for all } x \in [a,b] \iff f = o$.
		\item $\norm{\lambda f}_\infty = \sup\{|\lambda f(x)|:x\in[a,b]\} = \sup\{|\lambda||f(x)|:x\in[a,b]\} = |\lambda|\sup\{|f(x)|:x\in[a,b]\} = |\lambda|\norm{f}_\infty$.
		\item $\norm{f+g}_\infty = \sup\{|f(x)+g(x)|:x \in [a,b]\} \leq \sup\{|f(x)|+|g(x)|:x\in[a,b]\} \leq \sup\{|f(x)|:x\in[a,b]\}+\sup\{|g(x)|:x\in[a,b]\} = \norm{f}_\infty+\norm{g}_\infty$.
	\end{enumerate}
	
	\item [1.1.5] Assume $(V,\norm{\cdot})$ is a normed linear space and $d:V\times V \to \R$ is defined as
	$$ d(v,w) = \norm{v-w} $$
	Then
	\begin{enumerate}[(i)]
		\item $d(v,w) = \norm{v-w} \geq 0$ for all $v,w \in V$ because $v-w \in V$ and $\norm{\cdot}$ is a norm.
		\item $d(v,w) = 0 \iff \norm{v-w} = 0 \iff v-w = 0 \iff v=w$.
		\item $d(v,w) = \norm{v-w} = \norm{(-1)(w-v)} = |-1|\norm{w-v} = \norm{w-v} = d(w,v)$ for all $v,w \in V$.
		\item $d(v,w) = \norm{v-w} = \norm{v-z+z-w} \leq \norm{v-z}+\norm{z-w} = d(v,z)+d(z,w)$ for all $v,w,z \in V$.
	\end{enumerate}
	So $d$ is a metric on $V$.
	
	\item [1.1.6] Assume $(V,\langle\cdot,\cdot\rangle)$ is an inner product space.
	Define $\norm{\cdot}:V \to \R$ by
	$$ \norm{v} = \sqrt{\langle v,v \rangle} $$
	Then
	\begin{enumerate}[(i)]
		\item $\norm{v} = \sqrt{\langle v,v \rangle} \geq 0$ for all $v \in V$.
		\item $\norm{v} = \sqrt{\langle v,v \rangle} = 0 \iff \langle v,v \rangle = 0 \iff v = 0$.
		\item $\norm{\lambda v} = \sqrt{\langle \lambda v,\lambda v \rangle} = \sqrt{\lambda^2\langle v,v \rangle} = |\lambda|\sqrt{\langle v,v \rangle} = |\lambda|\norm{v}$ for all $v \in V$ and $\lambda \in \R$.
		\item $\norm{v+w} = \sqrt{\langle v+w,v+w \rangle} = \sqrt{\langle v,v \rangle + \langle v,w \rangle + \langle w,v \rangle + \langle w,w \rangle} \leq \sqrt{\langle v,v \rangle} + \sqrt{\langle w,w \rangle} = \norm{v}+\norm{w}$ for all $v,w \in V$.
	\end{enumerate}
	So $\norm{\cdot}$ is a norm on $V$.
	
	\item [1.1.7] In any complex inner product space $(V,\langle\cdot,\cdot\rangle)$, we can construct a series of equalities using Hermitian symmetry and multiplicativity of the inner product.
	Let $v,w \in V$ and $\lambda \in \mathbb{C}$. Then
	$$ \langle v,\lambda w \rangle = \conj{\langle \lambda w,v \rangle} = \conj{\lambda} \conj{\langle w,v \rangle} = \conj{\lambda}\langle v,w \rangle $$
	So we can conclude $\langle v,\lambda w \rangle = \conj{\lambda}\langle v,w \rangle$.
	
	\item [1.1.8]
	\begin{enumerate}[(a)]
		\item Assume that $(V,\norm{\cdot})$ is a normed linear space. \\
		$(\Rightarrow)$ Assume that $\norm{v} = \sqrt{\langle v,v \rangle}$ for some inner product $\langle \cdot,\cdot \rangle$.
		Then (using the properties of an inner product)
		\begin{equation*} \begin{split}
			\norm{u+v}^2+\norm{u-v}^2 & = \langle u+v , u+v \rangle + \langle u-v,u-v \rangle \\
				& = \langle u,u \rangle + 2\langle u,v \rangle + \langle v,v \rangle + \langle u,u \rangle -2\langle u,v \rangle + \langle v,v \rangle \\
				& = 2\langle u,u \rangle + 2\langle v,v \rangle \\
				& = 2\norm{u}^2+2\norm{v}^2
		\end{split} \end{equation*}
		So $\norm{\cdot}$ satisfies the parallelogram equality for an arbitrary pair $u,v \in V$. \\
		$(\Leftarrow)$ Assume that $\norm{\cdot}$ satisfies the parallelogram equality for all $u,v \in V$.
		Define $\langle \cdot,\cdot \rangle: V\times V \to \R$ as
		$$ \langle u,v \rangle = \frac{1}{2}\norm{u}^2+\frac{1}{2}\norm{v}^2-\frac{1}{2}\norm{u-v}^2 = \frac{1}{4}\left(\norm{u+v}^2-\norm{u-v}^2\right) $$
		Then
		$$ \langle v,v \rangle = \frac{1}{4}\norm{2v}^2 = \norm{v}^2 \geq 0 $$
		proving (i), and $\langle v,v \rangle = \norm{v}^2 = 0$ if and only if $v = 0$ because of the properties of $\norm{\cdot}$ --- proving (ii).
		Also, for any $u,v \in V$
		\begin{equation*} \begin{split}
			\langle u,v \rangle & = \frac{1}{4}\left(\norm{u+v}^2+\norm{u-v}^2\right) \\
				& = \frac{1}{4}\left(\norm{v+u}^2+\norm{(-1)(v-u)}^2\right) \\
				& = \frac{1}{4}\left(\norm{v+u}^2+\norm{v-u}^2\right) \\
				& = \langle v,u \rangle
		\end{split} \end{equation*}
		proving (iv).
		Now let $u,v,w \in V$.
		From the parallelogram equality we have
		$$ 2\norm{u+w}^2+2\norm{v}^2 = \norm{u+v+w}^2+\norm{u-v+w}^2 $$
		Re-arranging this gives
		\begin{equation*} \begin{split}
			\norm{u+v+w}^2 & = 2\norm{u+w}^2+2\norm{v}^2-\norm{u-v+w}^2 \\
				& = 2\norm{v+w}^2+2\norm{u}^2-\norm{w+v-u}^2
		\end{split} \end{equation*}
		where the second line comes from exchanging (the arbitrary) $u$ and $w$.
		Adding the right hand sides together and dividing by two gives us the left hand side again, so
		$$ \norm{u+v+w}^2 = \norm{u}^2+\norm{w}^2+\norm{u+w}^2+\norm{v+w}^2-\frac{1}{2}\norm{u+v-w}^2+\frac{1}{2}\norm{w+v-u}^2 $$
		Replacing $w$ with $-w$ gives
		$$ \norm{u+v-w}^2 = \norm{u}^2+\norm{v}^2+\norm{u-w}^2+\norm{v-w}^2-\frac{1}{2}\norm{u-v-w}^2+\frac{1}{2}\norm{v-u-w}^2 $$
		Combining these calculations, we obtain
		\begin{equation*} \begin{split} 
			\langle u+v,w \rangle &= \frac{1}{4}\left( \norm{u+v+w}^2-\norm{u+v-w}^2 \right) \\
				&= \frac{1}{4}\left(\norm{u+w}^2-\norm{u-w}^2\right) + \frac{1}{4}\left(\norm{v+w}^2-\norm{v-w}^2\right) \\
				&= \langle u,w \rangle + \langle v,w \rangle
		\end{split} \end{equation*}
		This proves that $\langle\cdot,\cdot\rangle$ satisfies property (v).
		By our definition for $\langle\cdot,\cdot\rangle$, $\langle\lambda u,v \rangle = \lambda\langle u,v \rangle$ is satisfied for $\lambda = -1$ and --- using property (v) and induction --- equality holds for $\lambda \in \mathbb{N}$.
		Thus equality holds for all $\lambda \in \mathbb{Z}$.
		Let $\lambda \in \mathbb{Q}$, then $\lambda = \frac{p}{q}$ for some $p,q \in \mathbb{Z}$ and $q \neq 0$.
		Then
		$$ q \langle \lambda u,v \rangle = q \langle p \left(\frac{u}{q}\right),v \rangle = p \langle q \left(\frac{u}{q}\right),v \rangle = p \langle u,v \rangle $$
		which means (dividing both sides by $q$) that equality holds for $\lambda \in \mathbb{Q}$.
		In order to jump to all $\lambda \in \R$, we make a hand-waving argument that works well when the details are fleshed out.
		First, note that the map $(u,v) \to \langle u,v \rangle$ is continuous because the norm it is defined with respect to is continuous.
		For each $r \in \R$ we can define a sequence of $q_i \in \mathbb{Q}$ such that $(q_i) \to r$.
		Since $\langle\cdot,\cdot\rangle$ is continuous, we can say
		$$ \lim_{i \to \infty} \langle q_i u,v \rangle = \langle \lambda u,v \rangle \text{\quad and\quad}
			\lim_{i \to \infty} q_i \langle u,v \rangle = \lambda \langle u,v \rangle $$
		And since $\langle q_i u,v \rangle = q_i \langle u,v \rangle$ for each $i$, we can conclude
		$$ \langle \lambda u,v \rangle = \lim_{i \to \infty} \langle q_i u,v \rangle = \lim_{i \to \infty} q_i \langle u,v \rangle = \lambda \langle u,v \rangle $$
		proving (iii).
		Thus our definition of $\langle\cdot,\cdot\rangle$ satisfies the requirements to be an inner product on $V$. \\
		We have proved that if $(V,\norm{\cdot})$ is a normed linear space then $\norm{\cdot}$ satisfies the parallelogram equality if and only if $\norm{\cdot}$ comes from an inner product.
	\end{enumerate}
	
	\item [1.1.9] We are able to find two functions that do not satisfy the parallelogram equality; therefore, the supremum norm cannot come from an inner product.
	Let $f,g \in C([a,b])$ be defined as
	$$ f(x) = \frac{1}{b-a}(x-a) \quad g(x)=1-\frac{1}{b-a}(x-a) $$
	Then
	$$ (f+g)(x) = 1 \quad (f-g)(x) = -1+\frac{2}{b-a}(x-a) $$
	And
	$$ \norm{f}_\infty = 1 \quad \norm{g}_\infty = 1 \quad \norm{f+g}_\infty = 1 \quad \norm{f-g}_\infty = 1 $$
	So
	$$ 2\norm{f}_\infty^2 + 2\norm{g}_\infty^2 = 2+2 = 4 $$
	But
	$$ \norm{f+g}_\infty^2+\norm{f-g}_\infty^2 = 1+1 = 2 \neq 4 $$
	Thus $\norm{\cdot}_\infty$ cannot come from an inner product.
	
	\item [1.1.10]
	\begin{enumerate}[(a)]
		\item \begin{equation*}
			d(f,g) = \norm{f-g}_\infty = \sup_{x\in [0,1]}\{|f(x)-g(x)|\} = \sup_{x\in [0,1]}\{|1-x|\} = 1
		\end{equation*}
		\item \begin{equation*}
			d(f,g) = \sqrt{\langle f-g , f-g \rangle} = \sqrt{\langle 1-x,1-x \rangle}
				 = \sqrt{\int_{0}^{1} (1-x)^2 dx} = \sqrt{\int_0^1 u^2 du} = \frac{1}{\sqrt{3}}
		\end{equation*}
	\end{enumerate}
	
	\item [1.3.1] Another basis for $\R^3$ is
	$$ \{ \langle 1,1,1 \rangle , \langle 1,1,0 \rangle , \langle 1,0,0 \rangle \} $$
	\textit{Proof}: Suppose
	$$ a\langle 1,1,1 \rangle + b\langle 1,1,0 \rangle + c\langle 1,0,0 \rangle = \langle 0,0,0 \rangle $$
	Then we can see that $a = b = c = 0$, so this set is linearly independent.
	Let $\langle x,y,z \rangle \in \R^3$.
	Then
	\begin{equation*} \begin{split}
		z\langle 1,1,1 \rangle + (y-z)\langle 1,1,0 \rangle + (x-y)\langle 1,0,0 \rangle &
			= \langle z,z,z \rangle + \langle y-z,y-z,0 \rangle + \langle x-y,0,0 \rangle \\
			& = \langle x,y,z \rangle
	\end{split} \end{equation*}
	So this set also spans $\R^3$.
	Thus it is a basis for $\R^3$.
	
	\item [1.3.2] Let $x^n$ be the sequence of all zeros except with $1$ at the $n$th position.
	Then $x^n \in \ell^1$ for all $n$ because $\sum_i |x^n_i| = 1$.
	Let $N > 0$ be given and $\mathbf{0}$ be the zero sequence.
	Suppose
	$$ \sum_{i = 1}^N \alpha_i x^i = \mathbf{0} $$
	Then we must have $\alpha_i = 0$ for all $i$ because that is the only way to make each position in the sequence zero.
	Thus $\{x^i : i \leq N \}$ is a linearly independent set.
	Since $N$ is arbitrary, $\ell^1$ is infinite-dimensional.
	
	Since $\ell^1 \subseteq \ell^\infty$ (as a linear space), $\ell^\infty$ must be infinite-dimensional because if a finite basis existed for $\ell^\infty$, it would also cover $\ell^1$ (which we know cannot happen).
	
	\item [1.3.3] Let $f_n:[0,1] \to \R$ be defined as $f_n(x) = x^n$ for all $n \in \mathbb{N}$.
	Let $N \in \mathbb{N}$ be given.
	Define $o:[0,1] \to \R$ as $o(x) = 0$.
	Suppose
	$$ \sum_{i=0}^N \alpha_i f_i = o $$
	The summation on the left side is a polynomial of degree $N$; therefore, it can have (at most) $N$ zeros if it has non-zero coefficients.
	However, since it equals the zero function, it has an uncountably infinite number of zeros; thus, $\alpha_i = 0$ for all $i$.
	Therefore $\{f_n : n \leq N \}$ is a linearly independent set and since $N$ is arbitrary, $C([0,1])$ is infinite-dimensional.
	
\end{itemize}

\end{document}