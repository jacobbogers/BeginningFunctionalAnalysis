\documentclass[../../Solutions.tex]{subfiles}

\begin{document}

\begin{itemize}

%	\item [4.1.1] This is a simple exercise of showing that the norm of each of these functions is $1$ and the inner product between any two distinct is zero.
%		Since the Lebesgue and Riemann integrals are equal for functions that are integrable in both theories, we calculate the Riemann integrals.
%		\begin{equation*} \begin{split}
%			\int_{-\pi}^{\pi} \left(\frac{1}{\sqrt{2\pi}}\right)^2 dx &= 1 \\
%			\int_{-\pi}^{\pi} \frac{\cos(nx)\cos(mx)}{\pi} dx &= \\
%			\int_{-\pi}^{\pi} \frac{\sin(nx)\sin(mx)}{\pi} dx &= \\
%			\int_{-\pi}^{\pi} \frac{\sin(nx)\cos(mx)}{\pi} dx &= 
%		\end{split} \end{equation*}	

	\item [4.1.2]
	\begin{enumerate}[(a)]
		\item We simply need to calculate the coefficients as defined at the beginning of section 4.1.
			\begin{equation*} \begin{split} 
				a_0 &= \frac{1}{\pi} \int_{-\pi}^{0} dx = 1 \\
				a_k &= \frac{1}{\pi} \int_{-\pi}^{0} \cos(kx) dx = \frac{1}{\pi}\left[\frac{1}{k}\sin(kx)\right]_{-\pi}^{0} = 0 \\
				b_k &= \frac{1}{\pi} \int_{-\pi}^{0} \sin(kx) dx = \frac{1}{\pi}\left[-\frac{1}{k}\cos(kx)\right]_{-\pi}^{0} = \frac{1}{\pi}\left(\frac{(-1)^k-1}{k}\right)
			\end{split} \end{equation*}
			Putting these calculations together with the expansion given, we have
			\begin{equation*}
				f(x) = \frac{1}{2}+\frac{1}{\pi}\sum_{k=1}^\infty \frac{(-1)^k-1}{k} \sin(kx)
			\end{equation*}
		\item %CONV IN MEAN
		\item $\sin(n0) = 0$ for all $n$, so the series gives a value of $1/2$ at $x = 0$, but $f(0)=0$. So this series does not converge pointwise.
			Since pointwise convergence implies uniform convergence, this series does not converge uniformly either.
		\item $f$ is a vertical shift (of $1/2$) from an odd function.
			All the cosine terms are even functions so they cannot contribute to making the series ``get closer" to $f$.
	\end{enumerate}
	
	\item [4.1.4]
	\begin{enumerate}[(a)]
		\item Assume $\{f_n\}_{n=1}^\infty$ converges to $f$ uniformly.
			Let $\epsilon > 0$ and $x \in [a,b]$ be given.
			Choose $N \in \mathbb{N}$ such that if $n \geq N$ and $y \in [a,b]$, then
			$$ |f_n(y)-f(y)| < \epsilon $$
			Thus we have (for a particular $y$)
			$$ |f_n(x)-f(x)| < \epsilon $$
			Therefore $\{f_n\}_{n=1}^\infty$ converges to $f$ pointwise.
		\item Define $\{f_n\}_{n=1}^\infty$ as
			$$ f_n(x) = \frac{1}{nx} $$
			This sequence converges to $f(x) = 0$ on $(0,1]$ pointwise (but not uniformly).
		\item Assume $\{f_n\}_{n=1}^\infty$ converges to $f$ uniformly.
			Let $\epsilon > 0$ be given.
			Choose $N \in \mathbb{N}$ such that if $n \geq N$ and $x \in [a,b]$ we have
			$$ |f_n(x)-f(x)| < \sqrt{\frac{\epsilon}{b-a}} $$
			Then
			$$ \int_{[a,b]} [f_n(x)-f(x)]^2 dm < \int_{[a,b]} \left(\sqrt{\frac{\epsilon}{b-a}}\right)^2 dm = \frac{\epsilon}{b-a}(b-a) = \epsilon $$
			Therefore $\{f_n\}_{n=1}^\infty$ converges to $f$ in mean.
		\item Define $\{f_n\}_{n=1}^\infty$ as
			\begin{equation*}
				f_n(x) = \begin{cases}
					n & x \in [-\frac{1}{n},0)\cup(0,\frac{1}{n}] \\
					1 & x = 0 \\
					0 & \text{otherwise}
				\end{cases}
			\end{equation*}
			This sequence converges pointwise to
			\begin{equation*}
				f_n(x) = \begin{cases}
					1 & x = 0 \\
					0 & \text{otherwise}
				\end{cases}
			\end{equation*}
			However
			\begin{equation*}
				\int_{[-1,1]} (f_n-f)^2 dm = \int_{[-1,1]} f_n^2 dm = 1
			\end{equation*}
			for all $n$, which does not go to zero as $n$ goes to infinity.
		\item Take the previous series from 4.1.2 as the counterexample.
			It converges in mean but not pointwise.
	\end{enumerate}
	
	\item [4.1.6]
	
	\item [4.2.1] Let $(V,\langle\cdot,\cdot\rangle)$ be an inner product space.
	\begin{enumerate}[(a)]
		\item Suppose $f,g \in V$ and that $f$ and $g$ are orthogonal ($\langle f,g \rangle = 0$).
			Then we have
			\begin{equation*} \begin{split} 
				\norm{f+g}^2 &= \langle f+g,f+g \rangle \\
					&= \langle f,f \rangle + \langle f,g \rangle + \langle g,f \rangle + \langle g,g \rangle \\
					&= \langle f,f \rangle + \langle g,g \rangle \\
					&= \norm{f}^2 + \norm{g}^2
			\end{split} \end{equation*}
		\item Assume that $\{f_k\}_{k=1}^\infty$ is an orthonormal sequence in $(V,\langle\cdot,\cdot\rangle)$.
			Let $f \in V$ be given. \\
			$(\Rightarrow)$ Suppose $\{f_k\}$ is complete.
			Then we write $f = \sum_{k=1}^\infty \langle f,f_k \rangle f_k$.
			Thus
			\begin{equation*} \begin{split} 
				\norm{f}^2 &= \norm{\sum_{k=1}^\infty \langle f,f_k \rangle f_k}^2 \\
					&= \sum_{k=1}^\infty \norm{\langle f,f_k \rangle f_k}^2 \\
					&= \sum_{k=1}^\infty |\langle f,f_k \rangle|^2 \norm{f_k}^2 \\
					&= \sum_{k=1}^{\infty} |\langle f,f_k \rangle|^2
			\end{split} \end{equation*}
			$(\Leftarrow)$ Now suppose $\norm{f}^2 = \sum_{k=1}^\infty |\langle f,f_k \rangle|^2$.
			Define $s_n = \sum_{k=1}^n \langle f,f_k \rangle f_k$.
			Then, using the argument given in the proof of Theorem 4.2, we have $\norm{f-s_n}^2 + \norm{s_n}^2 = \norm{f}^2$.
			Therefore
			\begin{equation*} \begin{split} 
				\lim_{n\to\infty} \norm{f-s_n}^2 &= \lim_{n\to\infty} \left(\norm{f}^2 - \norm{s_n}^2\right) \\
					&= \norm{f}^2 - \lim_{n\to\infty} \sum_{k=1}^n |\langle f,f_k \rangle| \\
					&= 0
			\end{split} \end{equation*}
			Thus we may write $f = \sum_{k=1}^\infty \langle f,f_k \rangle f_k$, showing that $\{f_k\}$ is complete.
	\end{enumerate}
	
	\item [4.2.3] $\{f_1\}$ is linearly independent ($c_1f_1 = 0 \iff c_1 = 0$ when $f_1 \neq 0$). \\
		Assume $\{f_i\}_{i=1}^{n-1}$ is linearly independent if it is orthonormal.
		Then suppose $\{f_i\}_{i=1}^n$ is orthonormal.
		Let $\sum_{i=1}^n c_if_i = 0$. Then
		$$ \left\langle \sum_{i=1}^n c_i f_i , f_n \right\rangle = c_n $$
		Thus $c_n$ must equal zero.
		Therefore, $\sum_{i=1}^{n-1} c_if_i = 0$ which means all of the $c_i$ are zero (by assumption).
		Thus $\{f_i\}_{i=1}^n$ is linearly independent (when orthonormal) for all $n$ by the Principle of Mathematical Induction.
	
	\item [4.2.4] This conclusion is shown a simple calculation.
		We point out that $\norm{g}^2 = \langle g,g \rangle$ and the definition of two vectors $h,k$ being orthogonal is $\langle h,k \rangle = 0$.
		\begin{equation*} \begin{split}
			\left\langle f-\frac{\langle f,g \rangle}{\langle g,g \rangle}g , \frac{\langle f,g \rangle}{\langle g,g \rangle}g \right\rangle
				&= \left\langle f, \frac{\langle f,g \rangle}{\langle g,g \rangle}g \right\rangle - \left\langle \frac{\langle f,g \rangle}{\langle g,g \rangle}g,\frac{\langle f,g \rangle}{\langle g,g \rangle}g \right\rangle \\
				&= \frac{\conj{\langle f,g \rangle}}{\conj{\langle g,g \rangle}} \langle f,g \rangle - \frac{\langle f,g \rangle \conj{\langle f,g \rangle}}{\langle g,g \rangle \conj{\langle g,g \rangle}}\langle g,g \rangle \\
				&= \frac{\langle f,g \rangle \conj{\langle f,g \rangle}}{\conj{\langle g,g \rangle}} - \frac{\langle f,g \rangle \conj{\langle f,g \rangle}}{\conj{\langle g,g \rangle}} \\
				&= 0
		\end{split} \end{equation*}
	
	\item [4.2.5]
	\begin{enumerate}[(a)]
		\item Since the function $f(x) = x$ is an odd function, none of the cosine terms will contribute.
			More specifically, $a_k = 0$ for all $k \in \mathbb{N}$.
			Now we calculate the $b_k$.
			\begin{equation*} \begin{split}
				b_k &= \frac{1}{\pi} \int_{-\pi}^{\pi} x\sin(kx) dx \\
					&= \frac{1}{\pi} \left(\left.-\frac{x}{k}\cos(kx)\right|_{-\pi}^{\pi} + \frac{1}{k}\int_{-\pi}^{\pi} \cos(kx) dx\right) \\
					&= \frac{1}{\pi}\left(-\frac{\pi}{k}\cos(k\pi)-\frac{\pi}{k}\cos(-k\pi)+0\right) \\
					&= 2\frac{(-1)^{k+1}}{k}
			\end{split} \end{equation*}
			Therefore, on $[-\pi,\pi]$, we have
			$$ x = 2\sum_{n=1}^\infty \frac{(-1)^{n+1}}{n}\sin(nx) $$
		\item Parsevals's identity states that $\norm{f}^2 = \sum_{n=1}^\infty |\langle f,f_n \rangle|^2$.
			In part (a) we have shown
			$$ |\langle f,f_n \rangle|^2 = \left|\frac{2(-1)^{n+1}}{n}\right|^2 = \frac{4}{n^2} $$
			And we can calculate $\norm{f}^2$
			$$ \norm{f}^2_2 = \int_{-\pi}^{\pi} x^2 dx = \left.\frac{x^3}{3}\right|_{-\pi}^{\pi} = \frac{2\pi^3}{3} $$
			Thus
			$$ \frac{2\pi^3}{3} = \sum_{n=1}^\infty \frac{4}{n^2} $$
			which is equivalent to
			$$ \sum_{n=1}^\infty \frac{1}{n^2} = \frac{\pi^3}{6} $$
	\end{enumerate}

\end{itemize}

\end{document}