\documentclass[../../Solutions.tex]{subfiles}

\begin{document}

\begin{itemize}
	\item [5.1.1] Let $H$ be a Hilbert space and $x_0 \in H$ be fixed.
		Define $T:H\to\mathbb{C}$ as
		$$ Tx = \langle x,x_0 \rangle $$
		Suppose $\alpha,\beta \in \mathbb{C}$ and $x,y \in H$. Then
		\begin{equation*} \begin{split} 
			T(\alpha x + \beta y) &= \langle \alpha x + \beta y , x_0 \rangle \\
				&= \langle \alpha x,x_0 \rangle + \langle \beta y, x_0 \rangle \\
				&= \alpha \langle x,x_0 \rangle + \beta \langle y, x_0 \rangle \\
				&= \alpha Tx + \beta Ty
		\end{split} \end{equation*}
		which shows that $T$ is a linear operator.
	
	\item [5.1.2] 
	\begin{enumerate}[(a)]
		\item Define $k:[0,b]\times[0,b]\to\mathbb{C}$ as
			\begin{equation*}
				k(s,t) = \begin{cases}
					0 & t>s \\
					\frac{1}{s} & t \leq s
				\end{cases}
			\end{equation*}
			Then $T:C([0,b])\to C([0,b])$ is
			$$ Tf(s) = \int_0^b k(s,t)f(t)dt = \int_0^s \frac{1}{s}f(t)dt = \frac{1}{s} \int_0^s f(t)dt $$
		\item Define $T$ as in part (a).
			Suppose $\alpha,\beta \in \mathbb{C}$ and $f,g \in C([0,b])$. Then
			\begin{equation*} \begin{split} 
				T(\alpha f + \beta g) &= \frac{1}{s} \int_0^s (\alpha f(t) + \beta g(t)) dt \\
					&= \alpha \frac{1}{s} \int_0^s f(t) dt + \beta \frac{1}{s} \int_0^s g(t) dt \\
					&= \alpha Tf + \beta Tg
			\end{split} \end{equation*}
	\end{enumerate}
	
	\item [5.1.3] Define the left shift $T:\ell^2\to\ell^2$ and right shift $S:\ell^2\to\ell^2$ as given in the text.
		Then
		$$ TS(a_1,a_2,a_3,\dots) = T(0,a_1,a_2,a_3,\dots) = (a_1,a_2,a_3,\dots) $$
		which shows that $TS = I$; however,
		$$ ST(a_1,a_2,a_3,\dots) = S(a_2,a_3,a_4,\dots) = (0,a_2,a_3,\dots) $$
		which shows that $ST \neq I$, and since $T$ is not onto, it cannot have a left inverse.
	
	\item [5.2.1] Let $(X,\norm{\cdot}_X)$ and $(Y,\norm{\cdot}_Y)$ be normed linear spaces.
		$\mathcal{B}(X,Y)$ is a linear space because if $\alpha,\beta \in \mathbb{C}$, $T,S \in \mathcal{B}(X,Y)$, and $x \in X$, then $T$ is bounded by some $M_T$ and $S$ by some $M_S$ and
		$$ (\alpha T + \beta S)x = \alpha Tx + \beta Sx \leq \alpha M_T x + \beta M_S x $$
		which means $\alpha T + \beta S \in \mathcal{B}(X,Y)$. \\
		Define $\norm{\cdot}_\mathcal{B}:\mathcal{B}(X,Y)\to\R$ as
		$$ \norm{T}_\mathcal{B} = \inf_{x\in X}\{M:\norm{Tx}_Y\leq M\norm{x}_X\} $$
		Then let $S,T \in \mathcal{B}(X,Y)$ and $\lambda \in \mathbb{C}$
		\begin{enumerate}[(i)]
			\item $\norm{T}_\mathcal{B} \geq 0$ because $\norm{Tx}_Y \geq 0$ for all $x \in X$.
			\item $\norm{T}_\mathcal{B} = 0$ if $\norm{Tx}_Y = 0$ for all $x$ where $\norm{x}_X \neq 0$.
				In other words, $Tx = 0$ for all $x$ with non zero norm, so $T$ is the zero operator almost everywhere (everywhere except where $\norm{x}_X = 0$).
			\item $\norm{\lambda T}_\mathcal{B} = \inf_{x\in X}\{M:\norm{\lambda Tx}_Y \leq M\norm{x}_X \}$ and we can pull out $\lambda$ because $\norm{\cdot}_Y$ is a norm. Thus $\norm{\lambda T}_\mathcal{B} = |\lambda|\inf_{x\in X}\{M:\norm{Tx}_Y \leq M\norm{x}_X \} = |\lambda|\norm{T}_\mathcal{B}$.
			\item This follows from the fact that $\norm{\cdot}_Y$ is a norm and thus follows the triangle inequality as well.
		\end{enumerate}
		Thus $\mathcal{B}(X,Y)$ is a normed linear space.
	
	\item [5.2.2] First, define $I:(C([0,1]),\norm{\cdot}_\infty)\to(C([0,1]),\norm{\cdot}_1)$ as $If = f$.
		Then
		$$ \norm{f}_1 = \int_{[0,1]} f dm \leq \int_{[0,1]} \norm{f}_\infty dm = \norm{f}_\infty $$
		Thus $\norm{If}_1 \leq 1\cdot\norm{f}_\infty$ which means $I$ is bounded. \\
		Contrarily, define $I:(C([0,1]),\norm{\cdot}_1)\to(C([0,1]),\norm{\cdot}_\infty)$ the same way.
		Then let $M \geq 1$ be given.
		Define $f \in C([0,1])$ as
		\begin{equation*}
			f(x) = \begin{cases}
				2M - 2M^2x & x \in [0,1/M] \\
				0 & \text{otherwise}
			\end{cases}
		\end{equation*}
		Then $\norm{f}_1 = 1$ and $\norm{f}_\infty = 2M$ which means $\norm{If}_\infty > M\norm{f}_1$.
		Thus $I$ is unbounded.
	
	\item [5.2.4] Let $a = \{a_n\}_{n=1}^\infty \in \ell^\infty$. Define $T:\ell^1\to\mathbb{C}$ as
		$$ Tx = \sum_{n=1}^\infty x_na_n $$
		Then
		\begin{equation*} \begin{split} 
			|Tx| &= \left|\sum_{n=1}^\infty x_n a_n\right| \\
				&\leq \sum_{n=1}^\infty |x_n a_n| \\
				&\leq \sum_{n=1}^\infty |x_n|\norm{a}_\infty \\
				&= \norm{a}_\infty \norm{x}_1
		\end{split} \end{equation*}
		Therefore $T$ is bounded.
		Define $a = (1,0,0,0,\dots)$ and $x = (1,0,0,0,\dots)$. Then
		$$ |Tx| = \left|\sum_{n=1}^\infty x_n a_n\right| = 1 = \norm{a}_\infty \norm{x}_1 $$
		Therefore $\norm{T} = \norm{a}_\infty$.
		
	\item [5.2.7]
	\begin{enumerate}[(a)]
		\item Since a differentiable function is necessarily continuous, $C^1([0,1]) \subseteq C([0,1])$.
			It is a subspace because the scalar multiplication of and the sum of two differentiable functions is a differentiable function.
			Define $f_n(x) = x^n$ for all $n$. These $f_n$ are in $C^1([0,1])$ but there pointwise limit is not in $C^1([0,1])$ (or event in $C([0,1])$).
		\item The differential operator is linear because if $\alpha,\beta \in \mathbb{C}$ and $f,g \in C^1([0,1])$ we have
		$$ \frac{d}{dx}\left[\alpha f + \beta g\right] = \alpha \frac{df}{dx} + \beta \frac{dg}{dx} $$
		However, define $g_n(x) = \sin(2\pi nx)$. Then $g_n \in C^([0,1])$ for all $n$ but
		$$ \frac{dg_n}{dx} = 2\pi n\cos(2\pi nx) \Longrightarrow \norm{\frac{dg_n}{dx}}_\infty = 2\pi n $$
		which is an unbounded sequence, so the differential operator is unbounded.
	\end{enumerate}
	
\end{itemize}

\end{document}